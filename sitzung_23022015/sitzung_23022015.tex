\documentclass[a4paper, 11pt]{article} % das Papierformat zuerst
\usepackage[utf8]{inputenc}
\usepackage{geometry}
\geometry{a4paper, top=25mm, left=35mm, right=25mm, bottom=25mm}
\usepackage{graphicx}
\usepackage{color}
\usepackage{epstopdf}
\usepackage[T1]{fontenc}
\usepackage{setspace}
\usepackage{tabularx}
\usepackage{blindtext}
\usepackage[ngerman]{babel} %deutsche Silbentrennung
\usepackage{titlesec}
\usepackage{enumitem} 
\usepackage{ifthen}

\definecolor{fsi}{RGB}{0,81,150}

\newcommand{\abstimmung}[4]{
	\newcounter{summe}
	\setcounter{summe}{#3}
	\addtocounter{summe}{#4}
	\begin{flushleft}
		#1\\
	Es wird über den Antrag abgestimmt.
	\end{flushleft}
	\ifthenelse{\equal{#3}{0}\AND\equal{#4}{0}}{
	\begin{center}
		\textbf{Der Antrag wird einstimmig angenommen.}
	\end{center}
	}{
	\begin{center}
		#2 \ifthenelse{\equal{#2}{1}}{Stimme}{Stimmen} dafür, #3 \ifthenelse{\equal{#3}{1}}{Stimme}{Stimmen} dagegen, #4 \ifthenelse{\equal{#42}{1}}{Enthaltung}{Enthaltungen}\\
		\ifthenelse{#2>\value{summe}}{
		\textbf{Der Antrag ist somit angenommen.}
		}{
		\textbf{Der Antrag ist somit abgelehnt.}
		}
	\end{center}
	}  
}


% VARIABLEN
\newcommand{\protokoller}{Elias Broschin}
\newcommand{\dateOfMeeting}{23. Februar 2015}
\newcommand{\TeXer}{Tobias Meier}
\newcommand{\fsiPresident}{Georg Schäfer}


\begin{document}
%deckblatt start

\doublespacing
\thispagestyle{empty}

\begin{center}
\includegraphics[width=10.0cm]{../logo_faculty_computer_science.eps}

\vspace*{\fill}
{\LARGE \textbf{Protokoll der Sitzung des Fachschaftsrates \\vom \dateOfMeeting}}\\
Fachschaftsrat Informatik\\
Trier University of Applied Sciences\\
\vspace{2.5cm}
\textit{
	Protokollführer: \textbf{\protokoller} \\
	\LaTeX - Umsetzung von \TeXer\\
	am \today
}
\vfill
\end{center}

\hspace*{-35cm}
\textcolor{fsi}{\rule{64.9cm}{15pt}}
\pagebreak
 
\setcounter{tocdepth}{2}
\tableofcontents 
\pagebreak

\section{Eröffnung}
Als Protokollführer wird \protokoller~bestimmt.\\
Der Sprecher des Fachschaftsrates \fsiPresident~eröffnet die Sitzung um 19:17 Uhr.
\\\\
%\textbf{Es wird festgestellt, dass der Fachschaftsrat vollzählig und beschlussfähig ist.}
%\\\\
%\\\\
\textbf{Es wird festgestellt, dass der Fachschaftsrat beschlussfähig ist.}\\
\textbf{Es fehlen:} Benjamin Albsmeier, Marco Lochen 
%\\\\
%% Das Protokoll der letzten Sitzung wurde verlesen und genehmigt.
%Die Genehmigung des Protokolls der letzten Sitzung wird auf die nächste Sitzung vertagt.

%\abstimmung{Es wird der Antrag gestellt, Dinge zu tun}{5}{6}{2} -- {dafür}{dagegen}{enthaltungen}
\section{Finanzen}
In der Ratssitzung am 14. Januar 2015, wurde beschlossen einen neuen Server anzuschaffen, hierzu wurde ein Antrag beim AStA eingereicht. Dieser Antrag wurde wegen Unklarheiten vom AStA abgelehnt! \\
Der Antrag wird vom Fachschaftsrat ergänzt und dem AStA zur nochmaligen Abstimmung übergeben. \\
\\
Zur Ausführung seiner Pflichten benötigt der Fachschaftsrat einen Drucker, deswegen wurde hierzu ein Antrag beim AStA gestellt. Dieser Antrag wurde vom AStA angenommen. \\
\\
Der Antrag auf einen 3D-Drucker wurde vom AStA abgelehnt. \\
\\
Es wurde über die Anschaffungsliste, welche man schon länger machen wollte (siehe Protokoll 14. Januar 2015), gesprochen. Da sich allerdings niemand damit befasst hat, ist man übereingekommen diese bis zur nächsten Sitzung nochmals zu ergänzen bzw. zu überarbeiten. \\

\section{Anschaffungen}

Die Sofas im Fachschaftsraum wurden notdürftig repariert. Da diese allerdings bald wieder kaputt gehen werden, will sich der Fachschaftsrat über die Kosten einer solchen Anschaffung informieren. \\
Es soll nach 2 Sofas mit folgenden Eigenschaften Ausschau gehalten werden: 
\begin{itemize}
	\item Bezug abziehbar und Waschmaschienen verträglich
	\item 3er Sofa \\
\end{itemize}
Die kleinen Tische im Fachschaftsraum sind auch nicht mehr die neusten und fallen stellenweise auseinander. Deswegen wurde vereinbart auch Angebote für kleine Tische herrauszusuchen. \\
Es wurde angemerkt, dass es schön wäre, wenn diese Tische runde Ecken hätten . \\
\\
Es wurde vereinbart, dass sich Philipp Dippel über die Angebote für die zwei Anschaffungen informiert. Die Ergebnisse dieser Suche werden dem restlichen Fachschaftsrat in der nächsten Sitzung mitgeteilt. \\
\\
Um 19:28 erscheint Benjamin Albsmeier zu Sitzung. \\

\section{Events}
\textbf{Bits don't Bite (BdB)} \\
Die neue Art der Durchführung der BdB hat nicht mehr Besucher als sonst gehabt. Es wurde gesagt, dass das Konzept zur Durchführung bis zur nächsten BdB überarbeitet werden muss. \\
\\
\textbf{Kneipentour} \\
Der Termin für die Kneipentour wurde mitgeteilt, sie findet am Mittwoch den 15.04.2015 statt. Eine Bekanntmachung des Termins per Mail steht noch aus.\\
Die Gruppenbildung soll, wie letztes Semester, auf die einzelnen Kneipen verteilt werden. Die Kneipen werden anschließend in einer festgelegten Reihenfolge abgelaufen. \\
Es wurde darüber gesprochen ein Doodle zu erzeugen. In diesem sollen sich intressierte Studentan ab dem 2. Semester eintragen, damit man die Teilnehmerzahl besser einschätzen kann. \\
\\
\textbf{GDW} \\
Die Räume für die Durchführung der GDW wurden reserviert. \\
Eine Genehmigung für einen Transponder, über den Zeitraum der GDW, wurde organisiert. Die Abholung beim Hausmeister steht noch aus. \\
Es wurde darüber informiert, dass 2 Studenten ihr Fachseminar in der GDW machen werden. \\
Das Bootcamp, welches am Freitag vor der GDW stattfindet, wurde organisiert. \\
Es wurde angemerkt, ob man nicht während der GDW einen Getränkeverkauf machen könnte. Es wird von der GDW-Orga geprüft, in welchem Umfang dies möglich ist und ob man dies noch zeitlich durch den AStA bringen kann. \\
\\
\textbf{Fussballtunier} \\
Das Fussballtunier soll nächstes Semester stattfinden, ansonsten kein Fortschritt. \\
\\
\textbf{100-Tage Grillen} \\
Im Rahmen des 100-Tage Projektes der Hochschule Trier soll es ein Grillen für die Erstis geben. Dieses wird von der Fachschaft organisiert und veranstaltet. Für die Finanzierung ist das 100-Tage Projekt verantwortlich. \\
Fleisch soll beim C+C Schaper eingekauft werden, Getränke bekommt man vom SWT.\\
\\
\textbf{Night of the living death (Notld)} \\
die Notld soll häufiger im Semester statfinden. Es wurde von drei bis zu vier festen Terminen im Semester geredet. Eine Organisation dieser Termine steht noch aus. \\

\section{Fachschaftsraum}
Der Putzplan hat wärend dem Semester gut funktioniert. Es sollte allerdings auch einen Putzplan für die Semesterferien erstellt und durchgeführt werden. \\
\\
Es wurde sich über ungespühlte und weitverteilte Tassen auf der Theke beschwert. \\
Desweiteren ist der Pizzaschneider immer noch nicht gespühlt worden. \\
\\
Der Kaffe im Fachschaftsraum ist abgelaufen. Es wurde überlegt, ob man diesen verschenken oder wegwerfen soll. Eine Entscheidung wurde nicht getroffen. \\
Es wurde auserdem angemerkt, dass die Kaffeemaschine im Fachschaftsraum nur Kaffee-Sülze macht. \\
\\
Stefan Bodenschatz ist um 19:59 der Sitzung beigetreten. \\
\\
Es wurde angemerkt, dass man die Pflanzen mehr giesen muss. \\

\section{Webseite}
Auf der Webseite sind zu wenige Updates zu Events, etc. \\
\\
Die News auf der Webseite sollen auch automatisch auf Twitter und Facebook gepostet werden. Dies soll in nächster Zeit realisiert werden. Wegen den Zugangsdaten kann Ramona angeschrieben werden. \\
\\
Marco soll mehr News schreiben. Hierrauf soll er hingewiesen werden. \\

\section{Sonstiges}
Aufgrund der Kritik am Modul Angewandte-Mathematik wurde eine Mail vom Fachschaftsrat an Herr. Rudolph geschickt. Im folgenden ist diese Mail, sowie die Antwortmail von Herr. Rudolph abgedruckt.
\begin{quote}
Sehr geehrter Herr Schäfer, \\
vielen Dank für Ihre Hinweise. \\
\\
Ich freue mich, dass Sie sich der Kritik angenommen haben.  Ich habe diese geprüft und bin zu der Überzeugung gelangt, dass die genannten  Punkte einer objektiven Betrachtung nicht standhalten. Die folgenden Punkte bitte ich nur als Hinweis, nicht als Entschuldigung zu verstehen:
\begin{enumerate}
\item die Übungsblätter sind nummeriert. Die Reihenfolge liegt also fest.
\item Die genannten zu wenig behandelten Verfahren wurden in der Vorlesung Schritt für Schritt vorgerechnet. Es existieren schriftliche Anleitungen. Deshalb wollten einige Übungsgruppen mehr Zeit für andere Aufgaben investieren. Diesem Wunsch bin ichnachgekommen.
\item Es existieren so viele Übungsaufgaben, dass die Zeit nicht dafür ausreichte.
\item Mit 25\% Durchfallquote sticht die Klausur nicht übermäßig hervor.
\item Schwierige Sonderfälle kamen in der Klausur nicht vor. 
\item Viele Aufgaben glichen denen der letzten Klausuren.
\end{enumerate}
Die Klausur wurde auch von Herrn Romes als besonders einfach klassifiziert. Wir sind beide der Überzeugung, dass wir diejenigen, welche in die Veranstaltungen gekommen sind, gut vorbereitet haben. Allerdings war die Beteiligung ab der Mitte des Semesters auch nicht besonders hoch. 108 Personen haben mitgeschrieben, doch nur ca. 50-60 kamen in die Übungen. In der letzten Vorlesung, in der typischerweise noch einmal Hinweise für die Klausur gegeben werden, kamen höchsten 75 Personen. \\
\\
Sie sehen also, dass dem Eindruck entgegen die Sachlage durchaus auch  völlig anders beurteilt werden kann.\\
\\
Ich bedanke mich für Ihre Mühe und \\
verbleibe mit freundlichen Grüßen \\
N. Rudolph \\
\\
-----Ursprüngliche Nachricht----- \\
Von: Georg Schäfer [mailto:schaefg@hochschule-trier.de] \\
Gesendet: Dienstag, 24. Februar 2015 14:56 \\
An: Prof. Dr. Fritz Nikolai Rudolph \\
Cc: Fachschaftsrat Informatik \\
Betreff: Kritik am Modul Angewandte-Mathematik \\
\\
Sehr geehrter Herr Prof. Dr. Rudolph, \\
\\
sicherlich ist Ihnen die Kritik an der vergangenen Angewandte-Mathematik  \\
Klausur nicht entgangen. Der Fachschaftsrat hat aus diesem Grund unter \\
den Studierenden eine Befragung durchgeführt, um den Gründen dieser \\
Kritik auf den Grund zu gehen. Dabei wurden die Studierenden gefragt, \\
wie sie mit Vorlesung, Übung und Klausur zurechtkamen. Im Folgenden \\
werden die wichtigsten Kritikpunkte genannt, die sich aus den erhaltenen \\
Rückmeldungen herauskristalisiert haben. \\
\\
Einige Studierende würden sich einen größeren Fokus auf klausurrelevante \\
Themen wünschen. Besonders stachen hier DLGs höherer Ordung und andere \\
Iterrationsverfahren als das Newton-Raphson-Verfahren heraus - Diese \\
Themen wurden wohl nicht ausreichend in der Veranstaltung behandelt. \\
Desweiteren wurde die Organisation der Übungsaufgaben angesprochen. \\
Scheinbar war oft nicht klar welche Übungsblätter zu bearbeiten waren \\
und der Stand der Übungsgruppen schien starke Diskrepanzen aufzuweisen. \\
Außerdem wurde bemängelt, dass zu einigen Themen zu wenige \\
Übungsaufgaben existieren bzw. diese in den Übungen nicht ausreichend \\
vertieft wurden. \\
\\
Wir hoffen dass Ihnen diese Kritik weiterhilft. \\
\\
Mit freundlichen Grüßen \\
Georg Schäfer \\
\\
-- \\ 
Georg Schäfer \\
Sprecher Fachschaftsrat Informatik \\
INF | INF \\
\end{quote}
\textbf{Die Sitzung wird um 20:19 geschlossen.} \\

\pagebreak
\section{Bestätigung des Protokolls}
Der Sprecher des Fachschaftsrates Informatik sowie der Protokollführer dieses Protokolls bestätigen mit Ihrer Unterschrift unter diesem Protokoll, dass selbiges inhaltlich korrekt ist und alle darin aufgeführten Beschlüsse so wie beschrieben vom Rat getragen werden.
\\

\vspace{3.5cm}
\hrulefill \hfill \hrulefill

\fsiPresident \hfill \TeXer

{\footnotesize (Sprecher)\hfill (Protokollführer)}
\end{document}
