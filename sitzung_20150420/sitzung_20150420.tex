

\documentclass[a4paper, 11pt]{article} % das Papierformat zuerst
\usepackage[utf8]{inputenc}
\usepackage{geometry}
\geometry{a4paper, top=25mm, left=35mm, right=25mm, bottom=25mm}
\usepackage{graphicx}
\usepackage{color}
\usepackage{epstopdf}
\usepackage[T1]{fontenc}
\usepackage{setspace}
\usepackage{tabularx}
\usepackage{blindtext}
\usepackage[ngerman]{babel} %deutsche Silbentrennung
\usepackage{titlesec}
\usepackage{enumitem} 
\usepackage{ifthen}

\definecolor{fsi}{RGB}{0,81,150}

\newcommand{\abstimmung}[4]{
	\newcounter{summe}
	\setcounter{summe}{#3}
	\addtocounter{summe}{#4}
	\begin{flushleft}
		#1\\
	Es wird über den Antrag abgestimmt.
	\end{flushleft}
	\ifthenelse{\equal{#3}{0}\AND\equal{#4}{0}}{
	\begin{center}
		\textbf{Der Antrag wird einstimmig angenommen.}
	\end{center}
	}{
	\begin{center}
		#2 \ifthenelse{\equal{#2}{1}}{Stimme}{Stimmen} dafür, #3 \ifthenelse{\equal{#3}{1}}{Stimme}{Stimmen} dagegen, #4 \ifthenelse{\equal{#42}{1}}{Enthaltung}{Enthaltungen}\\
		\ifthenelse{#2>\value{summe}}{
		\textbf{Der Antrag ist somit angenommen.}
		}{
		\textbf{Der Antrag ist somit abgelehnt.}
		}
	\end{center}
	}  
}


% VARIABLEN
\newcommand{\protokoller}{Tim Grundmanns}
\newcommand{\dateOfMeeting}{20. April 2015}
\newcommand{\TeXer}{Tim  Grundmanns}
\newcommand{\fsiPresident}{Georg Schäfer}


\begin{document}
%deckblatt start

\doublespacing
\thispagestyle{empty}

\begin{center}
\includegraphics[width=10.0cm]{../logo_faculty_computer_science.eps}

\vspace*{\fill}
{\LARGE \textbf{Protokoll der Sitzung des Fachschaftsrates \\vom \dateOfMeeting}}\\
Fachschaftsrat Informatik\\
Trier University of Applied Sciences\\
\vspace{2.5cm}
\textit{
	Protokollführer: \textbf{\protokoller} \\
	\LaTeX - Umsetzung von \TeXer\\
	am \today
}
\vfill
\end{center}

\hspace*{-35cm}
\textcolor{fsi}{\rule{64.9cm}{15pt}}
\pagebreak
 
\setcounter{tocdepth}{2}
\tableofcontents 
\pagebreak

\section{Eröffnung}
Als Protokollführer wird \protokoller~bestimmt.\\
Der Sprecher des Fachschaftsrates \fsiPresident~eröffnet die Sitzung um 19:22 Uhr.
\\\\
\textbf{Es wird festgestellt, dass der Fachschaftsrat vollzählig und beschlussfähig ist.}
\\\\
\pagebreak
\section{Anschaffungen}
\subsection{Server}
Am Mittwoch dem 22.April.2015, wird der Asta eine Sitzung halten und über den neuen Server der Fachschaft Informatik abstimmen.

Die Online-Umfrage ist wie folgt ausgefallen: 
Es gab insgesamt 121 Teilnehmer
11 stimmten gegen die Anschaffung eines Servers
26 waren sich unsicher, hielten einen Server jedoch für eine gute Idee
84 stimmten für einen neuen Server

Die Ergebnisse werden am oben genannten Sitzungstag dem Asta vorgelegt.

\subsection{Mobiliar}
Die von Philipp Dippel eingeholten Angebote sind immer noch aktuell, diese werden bis zur nächsten Sitzung vom Rat gesichtet werden um anschließend darüber abzustimmen.

\abstimmung{Es wird der Antrag gestellt, sieben neue Lacktische zu besorgen, identisch zu Jenen im Fachschaftsraum}{13}{0}{0}
Philipp Dippel erklärt sich bereit, den Transport des Mobiliars in die Wege zu leiten.

\subsection{Seperate Anschaffungen}
Kleinere Anschaffungen unter 50 Euro werden genehmigt und im folgenden besorgt.
Sämtliche Anschaffungen der Anschaffungsliste werden in der folgenden Woche als Antrag beim Asta eingereicht, da diese bereits bei der letzten Ratssitzung genehmigt worden sind.\\
Eine Ausnahme bildet hier die Anschaffung eines Bestecksets, welche aufgrund der Gefahr des Diebstahls abgelehnt wurde.\\
Zur Planung des bevorstehenden Fachschaftsgrillen wurde angemerkt, dass ein Heißluftfön von Vorteil für die kommenden Grillabende sei. Daher wird dieser ebenfalls beantragt.\\
Es sollte zudem ein Kalender angeschafft werden, der im Fachschaftsraum aufgehängt werden soll, um so die Einteilung von Putz- und Aufsichtsschichten zu erleichtern.
\pagebreak

\section{Events}
Die Begrüßung der Erstsemestler die Rundführung über das Hochschulgelände jedoch nur gering genutzt.

Die vom Fachschaftsrat organisierte Kneipentour ist ebenfalls erfolgreich verlaufen, Santo Pfingsten merkte jedoch an, dass man in Zukunft mehr Zeit für die  einzelnen Lokale einteilen sollte.

Im vergangenen Semester sind während der GameDevWeek entstanden. Die Präsentation für die Erstsemestler war jedoch nur geringfügig besucht.

Am 29.April.2015 wird erneut ein analoger Spieleabend stattfinden. Weitere Termine sind geplant.

Die Nacht der lebenden Entwickler wird am kommenden Freitag ab 20:00 stattfinden, Dieter Jahn wurde bereits nach der Mietung von Räumen gefragt, hat jedoch bisher noch nicht geantwortet.

Fabio Gimmillaro hat noch keinen Termin für das Fußballtunier eingeplant, wird sich jedoch demnächst nach dem Interesse der Studierenden erkundigen.

Das Fachschaftsgrillen ist ebenfalls noch in Planung. Michael Ochmann weist auf den erhöhten Bedarf von Kohle bei den Grillevents des Fachschaftsrates hin.

Die Netzwerkinformationsveranstaltung wurd für den Zeitraum des Juli angesetzt, genauere Informationen werden folgen.

Im Rahmen der Bits don't Bite soll es in Bälde wieder  zu Vorträgen kommen.

Michael Ochmann kritisiert die mangelnde Kommunikation zwischen den Ratsmitgliedern des Eventteams.
\pagebreak
\section{Fachschaftsraum}
Der Fachschaftsraum muss dringend gereinigt werden.

Ab sofort werden wieder die Zeiten für die Raumaufsicht, als auch die Putztermine eingeteilt.
Der Putzplan soll für das kommende Semester auch während der vorlesungsfreien Zeit fortgeführt werden, um den Raum sauber zu halten.

Georg Schäfer weist erneut daraufhin, dass nach wie vor benutzte und schmutzige Tassen im Fachschaftsraum stehen, es wird beschlossen, die Studierenden darauf hinzuweisen, dass diese Ihre Tassen reinigen oder aus dem Fachschaftsraum entfernen sollen, ansonsten werden diese durch den Fachschaftsrat entsorgt.

Es wird erneut festgestellt, dass Nicholas Gafford nach wie vor den Schlüssel des Fachschaftsschranks nicht zurückgebracht hat.
\pagebreak
\section{Finanzen}
Im Breich der Finanzen gab es keinerlei bemerkenswerte Änderungen.
\pagebreak
\section{Webseite}
Die Webseite funktioniert.
Michael Ochmann schlägt vor, die Startseite der Fachschaft neu zu gestalten um eine Überflutung von Informationen zu verhindern. Santo Pfingsten und Fabio Gimmillaro erklären sich bereit, bei dieser Neugestaltung zu helfen.

Ein Fehler der bei der Online-Ausleihe vorlag wurde immer noch nicht behoben und wird zunächst auch weiterhin bestehen.

Das Forum wurde auf unbestimmte Zeit  von der Webseite genommen, da es nicht genutzt wurde, der zuvor genannte Fehler ist vermutlich dadurch entstanden.

Michael Ochmann erklärt, dass er bereits an einer Lösung arbeite die bald zur Beseitigung des Problems führen sollte.

Die Own-Cloud, die während der GDW genutzt wurde, soll weiterhin genutzt werden können.  Der Registrierungsprozess soll zudem vereinfacht werden.

Santo Pfingsten schlägt die Einrichtung eines Lobbyservers vor.

20:13 Johannes Kirchner verlässt die Sitzung
\pagebreak
\section{Sonstiges}

Der Satzungsänderungsausschuss war erfolgreich.

20:15 Johannes Kirchner tritt der Sitzung bei

Die Satzungsänderung soll bis  Mitte nächster Woche für Verbesserungsvorschläge freigegeben werden und auf der kommenden Vollversammlung vorgestellt werden.

Die wichtigsten Änderungen betreffen die Wahl des Fachschaftsrates und Nachnominierungen, genauere Details sind der Satzungsänderung nachzulesen.

Michael Ochmann merkt an, dass der Zeitrahmen der Vollversammlung durch Präsentationen gefüllt werden sollte, da die Wahl des Fachschaftsrates bislang diese voll ausgefüllt hat.

Georg Schäfer schließt die Sitzung um 20:25.






%\\\\
%\textbf{Es wird festgestellt, dass der %Fachschaftsrat beschlussfähig ist.}\\
%\textbf{Es fehlen:} Karla Kollumna, Benjamin %Blümchen und Helge Schneider
%\\\\
%Das Protokoll der letzten Sitzung wurde verlesen und genehmigt.
%Die Genehmigung des Protokolls der letzten Sitzung wird auf die nächste Sitzung vertagt.

%\abstimmung{Es wird der Antrag gestellt, Dinge zu tun}{5}{6}{2} -- {dafür}{dagegen}{enthaltungen}

\pagebreak
\section{Bestätigung des Protokolls}
Der Sprecher des Fachschaftsrates Informatik sowie der Protokollführer dieses Protokolls bestätigen mit Ihrer Unterschrift unter diesem Protokoll, dass selbiges inhaltlich korrekt ist und alle darin aufgeführten Beschlüsse so wie beschrieben vom Rat getragen werden.
\\

\vspace{3.5cm}
\hrulefill \hfill \hrulefill

\fsiPresident \hfill \protokoller

{\footnotesize (Sprecher)\hfill (Protokollführer)}
\end{document}

    \documentclass[a4paper, 11pt]{article} % das Papierformat zuerst
\usepackage[utf8]{inputenc}
\usepackage{geometry}
\geometry{a4paper, top=25mm, left=35mm, right=25mm, bottom=25mm}
\usepackage{graphicx}
\usepackage{color}
\usepackage{epstopdf}
\usepackage[T1]{fontenc}
\usepackage{setspace}
\usepackage{tabularx}
\usepackage{blindtext}
\usepackage[ngerman]{babel} %deutsche Silbentrennung
\usepackage{titlesec}
\usepackage{enumitem} 
\usepackage{ifthen}

\definecolor{fsi}{RGB}{0,81,150}

\newcommand{\abstimmung}[4]{
	\newcounter{summe}
	\setcounter{summe}{#3}
	\addtocounter{summe}{#4}
	\begin{flushleft}
		#1\\
	Es wird über den Antrag abgestimmt.
	\end{flushleft}
	\ifthenelse{\equal{#3}{0}\AND\equal{#4}{0}}{
	\begin{center}
		\textbf{Der Antrag wird einstimmig angenommen.}
	\end{center}
	}{
	\begin{center}
		#2 \ifthenelse{\equal{#2}{1}}{Stimme}{Stimmen} dafür, #3 \ifthenelse{\equal{#3}{1}}{Stimme}{Stimmen} dagegen, #4 \ifthenelse{\equal{#42}{1}}{Enthaltung}{Enthaltungen}\\
		\ifthenelse{#2>\value{summe}}{
		\textbf{Der Antrag ist somit angenommen.}
		}{
		\textbf{Der Antrag ist somit abgelehnt.}
		}
	\end{center}
	}  
}


% VARIABLEN
\newcommand{\protokoller}{Max Mustermann}
\newcommand{\dateOfMeeting}{20. April 2015}
\newcommand{\TeXer}{Martina Mustermann}
\newcommand{\fsiPresident}{Georg Schäfer}


\begin{document}
%deckblatt start

\doublespacing
\thispagestyle{empty}

\begin{center}
\includegraphics[width=10.0cm]{../logo_faculty_computer_science.eps}

\vspace*{\fill}
{\LARGE \textbf{Protokoll der Sitzung des Fachschaftsrates \\vom \dateOfMeeting}}\\
Fachschaftsrat Informatik\\
Trier University of Applied Sciences\\
\vspace{2.5cm}
\textit{
	Protokollführer: \textbf{\protokoller} \\
	\LaTeX - Umsetzung von \TeXer\\
	am \today
}
\vfill
\end{center}

\hspace*{-35cm}
\textcolor{fsi}{\rule{64.9cm}{15pt}}
\pagebreak
 
\setcounter{tocdepth}{2}
\tableofcontents 
\pagebreak

\section{Eröffnung}
Als Protokollführer wird \protokoller~bestimmt.\\
Der Sprecher des Fachschaftsrates \fsiPresident~eröffnet die Sitzung um 14:33 Uhr.
\\\\
\textbf{Es wird festgestellt, dass der Fachschaftsrat vollzählig und beschlussfähig ist.}
\\\\
%\\\\
%\textbf{Es wird festgestellt, dass der %Fachschaftsrat beschlussfähig ist.}\\
%\textbf{Es fehlen:} Karla Kollumna, Benjamin %Blümchen und Helge Schneider
%\\\\
Das Protokoll der letzten Sitzung wurde verlesen und genehmigt.
%Die Genehmigung des Protokolls der letzten Sitzung wird auf die nächste Sitzung vertagt.

%\abstimmung{Es wird der Antrag gestellt, Dinge zu tun}{5}{6}{2} -- {dafür}{dagegen}{enthaltungen}

\pagebreak
\section{Bestätigung des Protokolls}
Der Sprecher des Fachschaftsrates Informatik sowie der Protokollführer dieses Protokolls bestätigen mit Ihrer Unterschrift unter diesem Protokoll, dass selbiges inhaltlich korrekt ist und alle darin aufgeführten Beschlüsse so wie beschrieben vom Rat getragen werden.
\\

\vspace{3.5cm}
\hrulefill \hfill \hrulefill

\fsiPresident \hfill \TeXer

{\footnotesize (Sprecher)\hfill (Protokollführer)}
\end{document}
