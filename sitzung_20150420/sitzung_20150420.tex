

\documentclass[a4paper, 11pt]{article} % das Papierformat zuerst
\usepackage[utf8]{inputenc}
\usepackage{geometry}
\geometry{a4paper, top=25mm, left=35mm, right=25mm, bottom=25mm}
\usepackage{graphicx}
\usepackage{color}
\usepackage{epstopdf}
\usepackage[T1]{fontenc}
\usepackage{setspace}
\usepackage{tabularx}
\usepackage{blindtext}
\usepackage[ngerman]{babel} %deutsche Silbentrennung
\usepackage{titlesec}
\usepackage{enumitem} 
\usepackage{ifthen}
\usepackage[colorlinks=true,linkcolor=fsi]{hyperref}

\definecolor{fsi}{RGB}{0,81,150}

\newcommand{\abstimmung}[4]{
	\newcounter{summe}
	\setcounter{summe}{#3}
	\addtocounter{summe}{#4}
	\begin{flushleft}
		#1\\
	Es wird über den Antrag abgestimmt.
	\end{flushleft}
	\ifthenelse{\equal{#3}{0}\AND\equal{#4}{0}}{
	\begin{center}
		\textbf{Der Antrag wird einstimmig angenommen.}
	\end{center}
	}{
	\begin{center}
		#2 \ifthenelse{\equal{#2}{1}}{Stimme}{Stimmen} dafür, #3 \ifthenelse{\equal{#3}{1}}{Stimme}{Stimmen} dagegen, #4 \ifthenelse{\equal{#42}{1}}{Enthaltung}{Enthaltungen}\\
		\ifthenelse{#2>\value{summe}}{
		\textbf{Der Antrag ist somit angenommen.}
		}{
		\textbf{Der Antrag ist somit abgelehnt.}
		}
	\end{center}
	}  
}


% VARIABLEN
\newcommand{\protokoller}{Tim Grundmanns}
\newcommand{\dateOfMeeting}{20. April 2015}
\newcommand{\TeXer}{Tim  Grundmanns}
\newcommand{\fsiPresident}{Georg Schäfer}


\begin{document}
%deckblatt start

\doublespacing
\thispagestyle{empty}

\begin{center}
\includegraphics[width=10.0cm]{../logo_faculty_computer_science.eps}

\vspace*{\fill}
{\LARGE \textbf{Protokoll der Sitzung des Fachschaftsrates \\vom \dateOfMeeting}}\\
Fachschaftsrat Informatik\\
Trier University of Applied Sciences\\
\vspace{2.5cm}
\textit{
	Protokollführer: \textbf{\protokoller} \\
	\LaTeX - Umsetzung von \TeXer\\
	am \today
}
\vfill
\end{center}

\hspace*{-35cm}
\textcolor{fsi}{\rule{64.9cm}{15pt}}
\pagebreak
 
\setcounter{tocdepth}{2}
\tableofcontents 
\pagebreak

\section{Eröffnung}
Als Protokollführer wird \protokoller~bestimmt.\\
Der Sprecher des Fachschaftsrates \fsiPresident~eröffnet die Sitzung um 19:22 Uhr.
\\\\
\textbf{Es wird festgestellt, dass der Fachschaftsrat vollzählig und beschlussfähig ist.}
\\\\
\section{Anschaffungen}
\subsection{Server}
Am Mittwoch den 22. April 2015, findet eine Sitzung des AStA statt, bei der über den neuen Server der Fachschaft Informatik abstimmt wird.\\
\begin{flushleft}
\textbf{Die Online-Umfrage zum Server ist zu folgendem Ergebnis gekommen:}\\
Es haben insgesamt 121 Studierende teilgenommen.\\
\textbf{11} Kommilitonen gaben an, den Server nicht nutzen zu wollen. \textbf{26} waren sich unsicher ob Sie die Dienste benötigen, hielten die Anschaffung jedoch für eine gute Idee. \textbf{84} Studierende gaben an, die angebotenen Dienste nutzen zu wollen.
\end{flushleft}

Die Ergebnisse werden am oben genannten Sitzungstag dem AStA vorgelegt.

\subsection{Mobiliar}
Die von Philipp Dippel eingeholten Angebote sind immer noch aktuell, diese werden bis zur nächsten Sitzung vom Rat gesichtet, damit im Anschluss darüber abgestimmt werden kann.

\abstimmung{Es wird der Antrag gestellt, sieben neue Tische des Fabrikats "`LACK"' \textit{(identisch der aktuellen)} anzuschaffen.}{13}{0}{0}
Philipp Dippel erklärt sich bereit, den Transport des Mobiliars in die Wege zu leiten.

\subsection{Seperate Anschaffungen}
Kleinere Anschaffungen im Wert \textbf{unter 50,00 Euro} werden genehmigt und im folgenden besorgt.
Sämtliche Anschaffungen der Anschaffungsliste werden in der folgenden Woche als Antrag beim AStA eingereicht, da diese bereits bei der letzten Ratssitzung genehmigt wurden.\\
Eine Ausnahme bildet hier die Anschaffung eines \textbf{Bestecksets}, welches aufgrund der Gefahr eines Diebstahls abgelehnt wurde.\\
Zur Planung des bevorstehenden Fachschaftsgrillen wurde angemerkt, dass ein \textbf{Heißluftfön} von Vorteil für die kommenden Grillabende sei. Daher wird dieser ebenfalls beantragt.\\
Es soll zudem ein Kalender angeschafft werden, der im Fachschaftsraum aufgehängt werden soll, um so die Einteilung von Putz- und Aufsichtsschichten zu erleichtern.

\section{Events}
\subsection{Erstsemesterbegrüßung}
Die bei der Begrüßung der Erstsemester angebotene Campusführung wurde nur spärlich besucht.

\subsection{Kneipentour}
Die vom Fachschaftsrat organisierte Kneipentour ist erfolgreich verlaufen, Santo Pfingsten merkte jedoch an, dass man in Zukunft mehr Zeit für die einzelnen Lokale einplanen soll.

\subsection{GameDevWeek}
Im vergangenen Semester sind während der GameDevWeek entstanden. Die Präsentation für die Erstsemestler war jedoch nur geringfügig besucht.

\subsection{Analogspieleabend}
Am 29. April 2015 wird ein weitere Analogspieleabend stattfinden. Weitere Termine sind geplant.

\subsection{Night of the living Devs}
Die Nacht der lebenden Entwickler wird am kommenden Freitag \textit{(24. April 2015)} ab 20:00 stattfinden. Dieter Jahn wurde bereits nach der Reservierung von Räumen gefragt, hat jedoch bisher noch nicht geantwortet.

\subsection{Fußballturnier}
Fabio Gimmillaro hat noch keinen Termin für das Fußballtunier eingeplant, wird sich jedoch demnächst nach dem Interesse der Studierenden erkundigen.

\subsection{Grillen}
Das Fachschaftsgrillen ist ebenfalls noch in Planung. Michael Ochmann weist auf den erhöhten Bedarf von Kohle bei den Grillevents des Fachschaftsrates hin.

\subsection{NIV}
Die Netzwerkinformationsveranstaltung wird für den Juli angesetzt, genauere Informationen werden folgen.

\subsection{Bits don't Bite}
Im Rahmen der Bits don't Bite sollen bald wieder Vorträge stattfinden.\\
\begin{flushleft}
	Michael Ochmann kritisiert die mangelnde Kommunikation zwischen den Eventmanagern des Fachschaftsrates.
\end{flushleft}

\section{Fachschaftsraum}
Der Fachschaftsraum muss dringend gereinigt werden.\\
Ab sofort werden wieder die Zeiten für die Raumaufsicht, als auch die Putztermine eingeteilt.
Der Putzplan soll für das kommende Semester auch während der vorlesungsfreien Zeit fortgeführt werden, um den Raum sauber zu halten.
\begin{flushleft}
Georg Schäfer weist erneut daraufhin, dass nach wie vor benutzte und schmutzige Tassen im Fachschaftsraum stehen, es wird beschlossen, die Studierenden darauf hinzuweisen, dass diese Ihre Tassen reinigen oder aus dem Fachschaftsraum entfernen sollen, ansonsten werden diese durch den Fachschaftsrat entsorgt.
\end{flushleft}
\begin{flushleft}
Es wird erneut festgestellt, dass Nicholas Gafford nach wie vor den Schlüssel des Fachschaftsschranks nicht zurückgebracht hat.
\end{flushleft}

\section{Finanzen}
Im Bereich Finanzen gab es keinerlei bemerkenswerte Änderungen.

\section{Webseite}
Die Webseite funktioniert.\\
Michael Ochmann schlägt vor, die Startseite der Webpräsenz neu zu gestalten um bessere Informationen direkt ausliefern zu können. Santo Pfingsten und Fabio Gimmillaro erklären sich bereit, bei dieser Neugestaltung zu helfen.\\
Ein Fehler der bei den Nutzerrechten vorlag wurde immer noch nicht behoben und wird zunächst auch weiterhin bestehen.\\
Das Forum wurde auf unbestimmte Zeit  von der Webseite genommen, da es nicht genutzt wurde. Der zuvor genannte Fehler wurde vermutlich vom Foren-Plugin verursacht.\\
Michael Ochmann erklärt, dass er bereits an einer Lösung arbeite die bald zur Beseitigung des Problems führen soll.
\begin{flushleft}
	Die ownCloud-Installation, die während der GDW genutzt wurde, soll weiterhin genutzt werden können. Der Registrierungsprozess soll zudem vereinfacht werden.
\end{flushleft}
\begin{flushleft}
	Santo Pfingsten schlägt die Einrichtung eines Lobbyservers vor.
\end{flushleft}

\begin{center}
\textit{20:13 Johannes Kirchner verlässt die Sitzung}
\end{center}

\pagebreak
\section{Sonstiges}
Der Satzungsänderungsausschuss war erfolgreich.

\begin{center}
\textit{20:15 Johannes Kirchner tritt der Sitzung bei}
\end{center}

\begin{flushleft}
Die Satzungsänderung soll bis  Mitte nächster Woche für Verbesserungsvorschläge freigegeben werden und auf der kommenden Vollversammlung vorgestellt werden.\\
Die wichtigsten Änderungen betreffen die Wahl des Fachschaftsrates und Nachnominierungen, genauere Details sind der Satzungsänderung nachzulesen.\\
Michael Ochmann merkt an, dass der Zeitrahmen der Vollversammlung durch Präsentationen gefüllt werden sollte, da die Wahl des Fachschaftsrates bislang diese voll ausgefüllt hat.
\end{flushleft}
Georg Schäfer schließt die Sitzung um 20:25.


\pagebreak
\section{Bestätigung des Protokolls}
Der Sprecher des Fachschaftsrates Informatik sowie der Protokollführer dieses Protokolls bestätigen mit Ihrer Unterschrift unter diesem Protokoll, dass selbiges inhaltlich korrekt ist und alle darin aufgeführten Beschlüsse so wie beschrieben vom Rat getragen werden.
\\

\vspace{3.5cm}
\hrulefill \hfill \hrulefill

\fsiPresident \hfill \protokoller

{\footnotesize (Sprecher)\hfill (Protokollführer)}
\end{document}