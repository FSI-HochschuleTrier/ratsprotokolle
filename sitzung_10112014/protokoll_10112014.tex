\documentclass[a4paper, 11pt]{article} % das Papierformat zuerst
\usepackage[utf8]{inputenc}
\usepackage{geometry}
\geometry{a4paper, top=25mm, left=35mm, right=25mm, bottom=25mm}
\usepackage{graphicx}
\usepackage{color}
\usepackage{epstopdf}
\usepackage[T1]{fontenc}
\usepackage{setspace}
\usepackage{tabularx}
\usepackage{blindtext}
\usepackage[ngerman]{babel} %deutsche Silbentrennung
\usepackage{titlesec}
\usepackage{enumitem} 

\definecolor{fsi}{RGB}{0,81,150}

% VARIABLEN
\newcommand{\protokoller}{Jonas Schön, Michael Ochmann}
\newcommand{\dateOfMeeting}{10. November 2014}
\newcommand{\TeXer}{Michael Ochmann}
\newcommand{\fsiPresident}{Santo Pfingsten}


\begin{document}
%deckblatt start

\doublespacing
\thispagestyle{empty}

\begin{center}
\includegraphics[width=10.0cm]{../logo_faculty_computer_science.eps}

\vspace*{\fill}
{\LARGE \textbf{Protokoll der Sitzung des Fachschaftsrates \\vom \dateOfMeeting}}\\
Fachschaftsrat Informatik\\
Trier University of Applied Sciences\\
\vspace{2.5cm}
\textit{
	Protokollführer: \textbf{\protokoller} \\
	\LaTeX - Umsetzung von \TeXer\\
	am \today
}
\vfill
\end{center}

\hspace*{-35cm}
\textcolor{fsi}{\rule{64.9cm}{15pt}}
\pagebreak
 
\setcounter{tocdepth}{2}
\tableofcontents 
\pagebreak

\section{Eröffnung}
Als Protokollführer wird Jonas Schön bestimmt.\\
Der Sprecher des ehemaligen Fachschaftsrates, \fsiPresident~eröffnet die Sitzung um 19:36 Uhr.
\\\\
\textbf{Es wird festgestellt, dass der Fachschaftsrat vollzählig und beschlussfähig ist.}
\subsection{Anwesende}
\begin{tabular}{ll}
	\textbf{gewählte, neue Ratsmitglieder} & \textbf{zusätzlich anwesende}\\
	Georg Schäfer & Markus Schmieder\\
	Santo Pfingsten & Konrad Puczynski\\
	Michael Ochmann & Jonas Schön\\
	Johannes Kirchner & Benjamin Albsmeier\\
	Thadeusz Wojcik & Lisa Schmidt\\
	Fabio Gimmillaro & Tobias Meier\\
	Tim Grundmanns & Dominik Petersdorf\\
	Elias Broschin & Robert Naumann\\
	 & Michael Gattinger\\
	 & Marco Lochen\\
	 & Philipp Dippel\\
	 & Professor Doktor Rainer Oechsle
\end{tabular}
\vspace{1.0cm}
\begin{flushleft}
Es wird festgestellt, dass der neue Rat die Amtsgeschäfte mit Beginn dieser Sitzung übernimmt.
\end{flushleft}

\section{Wahl der Hauptverantwortlichen}
Es wird der Antrag gestellt, zunächst die hauptverantwortlichen Ämter (sprich Sprecher, Kassenwart und ihre Stellvertreter) zu besetzen.

\begin{center}
	\textbf{Der Antrag wird einstimmig angenommen.}
\end{center}
Es wird vereinbart, die Besetzung dieser Ämter in geheimer Wahl durchzuführen.
Für das Amt des Sprechers stellen sich zur Wahl:
\begin{itemize}
	\item Georg Schäfer
	\item Thadeusz Wojcik
\end{itemize}
Für das Amt des Kassenwarts stellen sich zur Wahl:
\begin{itemize}
	\item Tim Grundmanns
	\item Fabio Gimmillaro
\end{itemize}
\vspace{1.0cm}
Georg Schäfer kann \textbf{sechs} Erststimmen auf sich vereinen, Thadeusz Wojcik \textbf{zwei}.\\
Damit ist Georg Schäfer der neue Sprecher des Fachschaftsrates Informatik und Thadeusz Wojcik sein Stellvertreter.
\begin{flushleft}
Tim Grundmanns kann \textbf{sieben} Erststimmen auf sich vereinen, Fabio Gimmillardo \textbf{eine}.
Damit ist Tim Grundmanns der neue Kassenwart der Fachschaft Informatik und Fabio Gimmillaro sein Stellvertreter.
\end{flushleft}

\begin{center}
	\textbf{Die Kandidaten nehmen die Wahl geschlossen an.}
\end{center}
Der Sprecher des Fachschaftsrates, Georg Schäfer übernimmt um 19:47 Uhr die Leitung der Sitzung.

\section{Nachnominierungen}
Die folgenden Personen bitten darum, nachträglich in den Rat aufgenommen zu werden. Dazu ist es erforderlich, dass der Kandidat weder eine 'Nein' Stimme erhält, noch die Anzahl der Enthaltungen die Anzahl der 'Ja' Stimmen übersteigt.\\
Um Aufnahme bitten folgende Personen:
\begin{flushleft}
	\textbf{Markus Schmieder, Philipp Dippel, Michael Gattinger, Benjamin Albsmeier, Tobias Meier, Dominik Petersdorf, Robert Naumann, Lisa Schmidt und Marco Lochen}
\end{flushleft}
\vspace{0.5cm}
Es wird der Antrag gestellt, die Abstimmung über die Nachnominierungen per Akklamation stattfinden zu lassen.
\begin{center}
	\textbf{Der Antrag wird abgelehnt.}
\end{center}
Es ergeht der Beschluss, dass die Abstimmung über die Nachnominierungen in geheimer Wahl stattfindet.
\subsection{Ergebnisse}
\begin{tabular}{|l|l|l|l|}
	\hline
	\textbf{Kandidat} & \textbf{'Ja' Stimmen} & \textbf{'Nein' Stimmen} & \textbf{Enthaltungen}\\ \hline
	Markus Schmieder & 3 & 2 & 3\\ \hline
	Philipp Dippel & 6 & 0 & 2\\ \hline
	Michael Gattinger & 3 & 2 & 3\\ \hline
	Benjamin Albsmeier & 8 & 0 & 0\\ \hline
	Tobias Meier & 7 & 0 & 1\\ \hline
	Dominik Petersdorf & 7 & 0 & 1\\ \hline
	Robert Naumann & 4 & 0 & 4\\ \hline
	Lisa Schmidt & 4 & 0 & 4\\ \hline
	Marco Lochen & 7 & 0 & 1\\
	\hline
\end{tabular}
\vspace{0.5cm}
\begin{flushleft}
	Die Anforderungen für die Aufnahme haben somit erfüllt:\\
	\textbf{Philipp Dippel, Benjamin Albsmeier, Tobias Meier, Dominik Petersdorf und Marco Lochen.}
\end{flushleft}
\begin{center}
	\textbf{Die Kandidaten nehmen die Wahl geschlossen an.}
\end{center}
Sie sind nun offiziell Mitglieder des Fachschaftsrates Informatik.

\section{Ansprache des Dekans}
Der Dekan des Fachbereichs Informatik an der Hochschule Trier, Herr Professor Doktor Rainer Oechsle ergreift das Wort.\\
Er vermittelt Informationen rund um das aktuelle Geschehen im Fachbereich, unter Anderem über den Hochschulentwicklungsplan, den neuen Studiengang Physiotherapie und die zukünftig geplanten Gesundheitsstudiengänge.\\\\
Der Sprecher unterbricht die Sitzung um \textbf{20:53 Uhr}.\\\\
Die Sitzung wird um \textbf{21:04 Uhr} fortgesetzt. Es haben die Sitzung verlassen:
\begin{flushleft}
	\textbf{Michael Gattinger und Markus Schmieder.}
\end{flushleft}
Michael Ochmann führt das Protokoll weiter.

\section{Anliegen Außenstehender}
Der Präsident des Studierendenparlaments, Jonas Schön gibt zu Protokoll, dass die neue Finanzordnung der Studierendenschaft der Hochschule Trier in kraft getreten ist. Er bittet weiter um Beteiligung bei der Änderung der Satzung der Studierendenschaft, mit der sich derzeit ein Ausschuss bestehend aus Mitgliedern des Studierendenparlamentes und Studierenden der Hochschule beschäftigt.\\\\
Der Sprecher des Allgemeinen Studierendenausschusses der Hochschule Trier, Konrad Puczynski, stellt den Antrag, den alten Fachschaftsrat zu entlasten.
\begin{center}
	\textbf{Der Antrag wird einstimmig abgelehnt. Es ergeht der Beschluss, die Entscheidungen und Vorgänge des alten Rates zu prüfen und demnach die Entlastung zu vertagen.}
\end{center}
Es verlassen die Sitzung:
\begin{flushleft}
	\textbf{Jonas Schön und Konrad Puczynski}
\end{flushleft}

\section{Weitere Anträge}
Fabio Gimmillaro teilt mit, dass er von seinem Amt als stellvertretender Kassenwart zurücktritt.\\
Es muss ein neuer stellvertretender Kassenwart gewählt werden.\\ 
Georg Schäfer stellt den Antrag, die Wahl für den neuen stellvertretenden Kassenwart per Akklamation stattfinden zu lassen.
\begin{center}
	\textbf{Der Antrag wird einstimmig angenommen.}
\end{center}
Zur Wahl stellt sich \textbf{Philipp Dippel}.\\
\begin{center}
	\textbf{Philipp Dippel wird einstimmig als stellvertretender Kassenwart bestätigt.}
\end{center}

\section{Wahl der weiteren Ämter}
Sprecher Georg Schäfer und die anwesenden Mitglieder des alten Rates geben einen Überblick über die Aufgabenbereiche der einzelnen Ämter.
Es finden die Wahlen für die Vergabe der verbliebenen Ämter statt.

\subsection{Systemadministrator}
Zur Wahl stellt sich Johannes Kirchner.
\begin{center}
	\textbf{Johannes Kirchner wird einstimmig als Systemadministrator bestätigt.}
\end{center}

\subsection{Webmaster}
Zur Wahl stellen sich Santo Pfingsten und Michael Ochmann.\\
Santo Pfingsten erhält \textbf{keine} Stimme, Michael Ochmann \textbf{zwölf}.
\begin{center}
	\textbf{Michael Ochmann ist somit der neue Webmaster der Fachschaft.}
\end{center} 

\subsection{Raumwart / Klausurminister}
Zur Wahl stellen sich Benjamin Albsmeier, Elias Broschin und Tobias Meier.\\
Benjamin Albsmeier erhält \textbf{fünf} Stimmen, Elias Broschin \textbf{zwei} und Tobias Meier \textbf{vier} Stimmen.
\begin{center}
	\textbf{Benjamin Albsmeier ist somit der neue Raumwart / Klausurminister der Fachschaft}
\end{center}

\subsection{Pressesprecher}
Zu Wahl stellen sich Fabio Gimmillaro und Marco Lochen.\\
Fabio Gimmillaro erhält \textbf{fünf} Stimmen, Marco Lochen \textbf{sieben} Stimmen.
\begin{center}
	\textbf{Marco Lochen ist somit der neue Pressesprecher der Fachschaft.}
\end{center}

\subsection{Eventmanager}
Da zur Zeit die Ämter 'Zeugwart' und 'Getränkewart' aufgrund des Finanzembargos nicht besetzt werden müssen, wird beschlossen alle verbliebenen Ratsmitglieder als Eventmanager einzusetzen, jedoch einen Hauptverantwortlichen zu wählen.\\
Zur Wahl stehen Santo Pfingsten und Fabio Gimmillaro.\\
Santo Pfingsten erhält \textbf{elf} Stimmen, Fabion Gimmillaro erhält \textbf{zwei} Stimmen.
\begin{center}
	\textbf{Santo Pfingsten ist somit Verantwortlicher für Eventmanagement.}
\end{center}
Elias Broschin, Tobias Meier, Fabio Gimmilaro und Dominik Petersdorf sind weiter auch als Eventmanager eingesetzt.

\begin{center}
	\textbf{Alle Kandidaten nehmen die Wahl an.}
\end{center}
Michael Ochmann stellt den Antrag, jedem Amt eine Emailadresse auf der Domain der Fachschaft (fsi.hochschule-trier.de) zuzuordnen.
\begin{center}
	\textbf{Der Antrag wird einstimmig angenommen.}
\end{center}
Der Sprecher Georg Schäfer schließt die Sitzung um 22:02 Uhr.

\pagebreak
\section{Bestätigung des Protokolls}
Alle Mitglieder des Fachschaftsrates Informatik bestätigen mit Ihrer Unterschrift unter diesem Protokoll, dass selbiges inhaltlich korrekt ist und alle darin aufgeführten Beschlüsse so wie beschrieben vom Rat getragen werden.
\\
\vspace{1.5cm}

\hrulefill \hfill \hrulefill

Elias Broschin \hfill Tim Grundmanns

\vspace{2.0cm}
\hrulefill \hfill \hrulefill

Fabio Gimmillaro \hfill Santo Pfingsten

\vspace{2.0cm}
\hrulefill \hfill \hrulefill

Johannes Kirchner \hfill Michael Ochmann

\pagebreak
\vspace{2.0cm}
\hrulefill \hfill \hrulefill

Philipp Dippel \hfill Benjamin Albsmeier

\vspace{2.0cm}
\hrulefill \hfill \hrulefill

Marco Lochen \hfill Tobias Meier

\vspace{2.0cm}
\hrulefill \hfill \hfill \hfill

Dominik Petersdorf \hfill \hfill

\vspace{3.5cm}
\hrulefill \hfill \hrulefill

Georg Schäfer \hfill Thadeusz Wojcik

{\footnotesize (Sprecher)\hfill (stellv. Sprecher)}

\end{document}
