\documentclass{hst-protokoll} % eigene Dokumentenklasse

% VARIABLEN
\ProtokollID      {\#10}
\Protokollfuehrer {Michael Ochmann}
\Sitzungsdatum    {12. Oktober 2015}
\LatexVon         {Michael Ochmann}
\Vorsitzender     {Georg Schäfer}
\Logo             {logo_faculty_computer_science.eps}
\AkzentFarbe      {0}{81}{150}

\begin{protokoll}

\section{Eröffnung}
Als Protokollführer wird \protokoller~bestimmt.\\
Der Sprecher des Fachschaftsrates \fsiPresident~eröffnet die Sitzung um 19:37 Uhr.
\\\\
\textbf{Es wird festgestellt, dass der Fachschaftsrat beschlussfähig ist.}\\
\textbf{Es fehlen:} Fabio Gimmillaro und Tobias Meier
\\\\
Das Protokoll der letzten Sitzung wurde verlesen, berichtigt und genehmigt.

\section{Finanzen}
- Finanz-jahresbericht beim asta angefordert

\section{Events}
santo sagt gdw gut. sponsoring gut. drei spiele entstanden. bits dont bite über gdw erlebnisse. esti begrüßung gut verlaufen.
kneipentour geplant für do 22.10.2015. genügend kneipen und führer [kneipen auflisten]

georg sagt eventleute sind kake. santo schlägt trello vor und ist für vorschläge offen. mike sagt zuständigkeit ist behindert.

gespräche übers grillen (stefan fragt). überlegt kleines "inoffizielles grillen" zu veranstalten (während des glühens eventuell)

santo spricht fußballturnier an. frage ob abschaffen. georg dagegen. allgemeinere sportveranstaltungen ins auge fassen. lisa schmidt fragt wie viele studierende zum zustandekommen benötigt werden. (eine mannschaft (11))

mike -> events für nicht-geeks. muss überlegt werden

\section{Anschaffungen}
nicht mehr so viel geld da. sofas wichtiger als schrank. fachschaftsgeld kann beantragt werden. (1600 noch da, aber noch rechnungen offen)

neues Maskottchen muss her. vorschlag wheatley aus portal2. muss diskutiert werden. stefan und santo sorge wg markenrrechtsverletzung.

mike -> sammelantrag wegen bürostuff (liste)

\section{Wahl zum Fachschaftsrat 2015}
wahlausschuss hat sich letzte woche zum ersten mal getroffen
mitglieder:::
wahltermin vorschlag 22 \& 23. 10. (freitag wg gesundheitsleuten). Der Raum F3 wurde reserviert. ticket läuft für mailingliste für wahlausschuss (wahlen@fsi)

überlegungen wie kommillitonen zu informieren über arbeit des fsri. allgemein schlechte beteiligung wird festgestellt.

\section{Vollversammlung}
muss vor der konsti sitzung vom neuen rat stattfinden. günstiger zeitrahmen muss gefunden werden um mögl viele leute zu erreichen. wieder mittagspause, da abends niemand kommt.

anständige präsi machen.

dippel erklärt sich bereit, präsi zu machen. jeder soll dich gedanken machen, was er über seine arbeit im letzten jahr sagen kann.

\section{Sonstiges}

\subsection{Austritte}
dominik und tim 

\subsection{Stromanschluss draußen}
mike erzählt von arnes geschichte.
georg will bei föhr nachfragen

\subsection{Fahrradständer}
georg will bei oechsle nachfragen

\subsection{renovierung hs3}
santo weißt hin dass dieses semester noch renoviert wird. veranstaltungen verlegt in aula.

\subsection{alte lack tische}
müssen weg

\subsection{Pi and More}
fsri soll nächste pi \& more (sa 16. januar) an der hst veranstalten. räume sind beantragt, warten auf repsonse.

ausschuss soll gegründet werden, wg unabhängigkeit vom fsri wg neuwahlen etc.

\abstimmung{Es wird der Antrag gestellt, einen Ausschuss zu bilden der mit entsprechenden Vollmachten ausgestattet ist um unabhängig vom amtierenden FSRI die Veranstaltung zu organisieren.}{8}{0}{0}

Beschluss \#FSRI\_4 dass das so ist.

(Mitglieder mike, georg, santo)

pis sollen organisiert werden. 
Pies sollen organisiert werden (kuchen)[lustige witz von santo]

\subsection{Raum}
putzplan funzt nicht. plan abgelaufen. albsi neu machen?

tische leicht lädiert.

santo bemerkt dreckiges besteck und müll.

raumzugang auf hp beantragen. 

\abstimmung{Es wird der Antrag gestellt, die LACK Tische zu entsorgen}{8}{0}{0}
Es ergeht der Beschluss \textit{(\#FSRI\_3)} dass die LACK Tische entsorgt werden.

\subsection{GDW Orga}
mike -> fachseminarteilnehmer als haupt orga leute ungeeignet. jemand "greifbares" mit feuer und flamme.
kritik an santo per mail.

Georg beendet die sitzung um 20:58 Uhr.

% % ENDBLATT
\pagebreak
\section{Bestätigung des Protokolls}
Der Sprecher des Fachschaftsrates Informatik sowie der Protokollführer dieses Protokolls bestätigen mit Ihrer Unterschrift unter diesem Protokoll, dass selbiges inhaltlich korrekt ist und alle darin aufgeführten Beschlüsse so wie beschrieben vom Rat getragen werden.
\\

\vspace{3.5cm}
\hrulefill \hfill \hrulefill

\fsiPresident \hfill \protokoller

{\footnotesize (Sprecher)\hfill (Protokollführer)}
\end{protokoll}
