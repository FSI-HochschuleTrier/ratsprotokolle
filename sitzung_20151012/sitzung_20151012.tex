\documentclass{hst-protokoll} % eigene Dokumentenklasse
\usepackage{multicol}

% VARIABLEN
\ProtokollID      {\#10}
\Protokollfuehrer {Michael Ochmann}
\Sitzungsdatum    {12. Oktober 2015}
\LatexVon         {Michael Ochmann}
\Vorsitzender     {Georg Schäfer}
\Logo             {logo_faculty_computer_science.eps}
\AkzentFarbe      {0}{81}{150}

\begin{protokoll}

\section{Eröffnung}
Als Protokollführer wird \protokoller~bestimmt.\\
Der Sprecher des Fachschaftsrates \fsiPresident~eröffnet die Sitzung um 19:37 Uhr.
\\\\
\textbf{Es wird festgestellt, dass der Fachschaftsrat beschlussfähig ist.}\\
\textbf{Es fehlen:} Fabio Gimmillaro und Tobias Meier
\\\\
Das Protokoll der letzten Sitzung wurde verlesen, berichtigt und genehmigt.

\section{Finanzen}
Der Jahresbericht über die Finanzen des Fachschaftsrates Informatik wurde beim Allgemeinen Studierendenausschuss angefordert. Dieser wird benötigt, damit der Fachschaftsrat bei der Vollversammlung der Fachschaft Informatik der Hochschule Trier Rechenschaft ablegen kann.

\section{Events}
Santo Pfingsten berichtet, dass die GameDevelopmentWeek für das Sommersemester 2015 insgesamt gut verlaufen ist. Es kamen insgesamt drei Spiele zustande, zwei in der Java-Gruppe und eins in der C++-Gruppe. Auch das Sponsoring dass stattgefunden hat ist ohne Probleme abgelaufen und kam sowohl bei den Teilnehmern, als auch den Sponsoren gut an. Er führt weiter aus, dass die nächste Bits don't Bite Veranstaltung sich hauptsächlich mit Erlebnissen und den Ergebnissen der GameDevWeek beschäftigen soll.\\
Santo berichtet, dass die Erstibegrüßung ohne größere Probleme verlaufen ist.\\
Die Kneipentour für das Wintersemester 2015/2016 ist für den 22. Oktober 2015 geplant. Es sind genügend Gruppenleiter für die voraussichtlich große Zahl der Teilnehmer vorhanden. Besucht werden die Kneipen

\begin{multicols}{2}
	\begin{itemize}
		\singlespacing
		\item Cubiculum
		\item Havanna
		\item Astarix
		\item Irish Pub
		\item New Mintons
		\item Styxx
		\item Cheers
		\item Zapotex
	\end{itemize}
\end{multicols}
\noindent
Georg Schäfer führt abermals die schlechte Koordination unter den Eventmanagern an. Santo Pfingsten entgegnet, dass er den Managementdienst "`Trello"' zum Einsatz bringen möchte, aber keiner der anderen Eventmanager mitzieht. Er ist offen für Vorschläge zur Verbesserung der Koordination. Michael Ochmann bemängelt im Speziellen die strikte Zuordnung von Zuständigkeiten für verschiedene Events unter den Eventmanagern. Trotz dieser Aufteilung müsse jeder Eventmanager über alle Events bescheid wissen und Auskunft geben können. Santo Pfingsten nimmt die zur Kenntnis.\\
Das Fachschaftsgrillen kommt zur Sprache, da Stefan Bodenschatz um Informationen dazu bittet. Das Grillen fällt unter den Zuständigkeitsbereich von Tobias Meier, da dieser nicht anwesend ist, kann nicht festgestellt werden, warum das Grillen nicht stattgefunden hat. Es werden Überlegungen angestellt, eventuell ein kleines, "`inoffizielles"' Fachschaftsgrillen zu organisieren. Es wird vorgeschlagen, dies während des Glühen-Events zu tun.\\
Santo Pfingsten bringt das Fußballturnier zur Sprache. Er stellt die Frage, ob das Turnier aus der Liste der regelmäßigen Events des Fachschaftsrates gestrichen werden soll. Georg Schäfer spricht sich entschieden dagegen aus, da das Fußballturnier sowohl bei den Studierenden, als auch dem Lehrkörper als Traditionsveranstaltung gehandelt wird. Statt es zu streichen, soll das Turnier wiederbelebt werden. Lisa Schmidt stellt die Frage, wie viele Studierende zum Zustandekommen des Turniers benötigt würden. Es wird festgestellt, dass mindestens eine Mannschaft von elf Personen benötigt wird.\\
Es wird weiter überlegt, auch allgemeinere Sportveranstaltungen ins Auge fassen.\\
Michael Ochmann teilt dem Rat mit, dass er bereits darauf angesprochen wurde, dass der Fachschaftsrat wenige regelmäßige Events für "`Nicht-Geeks"' veranstaltet. Der Rat entgegnet, dass sowohl das Fußballturnier, als auch andere Veranstaltungen wie z.B. das Glühen, die Ersti-Kneipentour oder die Bits don't Bite keine "`Geek-Events"' sind. Es soll aber in der Zukunft noch einmal zur Sprache gebracht werden.

\section{Anschaffungen}
Es wird festgestellt, dass der Fachschaft Informatik nicht mehr so viel Geld zur Verfügung steht, wie angenommen. der anwesende AStA-Sprecher Dominik Petersdorf teilt mit, dass auf dem Konto der Fachschaft aktuell noch ca. 1600,00 \texteuro stehen, aber noch nicht alle offenen Rechnungen bezahlt sind, da die Finanzreferentin des ehemaligen AStA, Finja Risse seit Januar 2015 nur sporadisch Rechnungen bezahlt hat. Diese Feststellung führt zu der Annahme, dass in diesem Semester erstmals wieder Fachschaftsgeld beantragt werden kann. Da noch neue Sofas und ein neuer Schrank auf der Liste der nötigen Anschaffungen stehen wird festgestellt, dass neue Sofas definitiv Priorität gegenüber eines neuen Aktenschrankes haben.\\
Da das alte Maskottchen der Fachschaft, "`Heinz die Lampe"' mittlerweile völlig zerstört und irreparabel ist, muss ein neues Maskottchen her.\\
Michael Ochmann schlägt eine mechatronische Version des Charakters Wheatly aus dem Spiel "`Portal 2"' vor, da in dem Projekt sowohl handwerkliche, programmatische als auch elektrotechnische Probleme gelöst werden müssen. Georg Schäfer stimmt dem zu und merkt an, dass auch die Dokumentation der Herstellung für viele Menschen interessant sein könnte.\\
Stefan Bodenschatz und Santo Pfingsten melden Bedenken wegen Markenrechtsverletzungen an.\\
Michael Ochmann teilt mit, dass ein Sammelantrag für Verbrauchsmaterialien im Bürobereich nötig ist.

\section{Wahl zum Fachschaftsrat 2015}
Der Wahlausschuss für die Wahl zum Fachschaftsrat in der Legislaturperiode 2015/2016, bestehend aus Dominik Petersdorf, Tim Grundmanns, Thaddeusz Wojcik und [\ldots] hat in der letzten Woche erstmals getagt.\\
Als Termin für die Wahl wird der 22. \& 23. Oktober 2015 vorgeschlagen. Hintergrund für diesen Termin sind die Kommilitonen in den Gesundheitsstudiengängen, die nur Freitags und Samstags am Campus anwesend sind.\\
Der Raum F3 wurde als Wahllokal reserviert.\\
Michael Ochmann teilt mit, dass bereits ein Ticket beim Rechenzentrum der Hochschule eröffnet wurde, um einen Mailverteiler für den Wahlauschuss Informatik (wahlen@fsi.hochschule-trier.de) einzurichten.\\
Es werden Überlegungen angestellt, wie man die Kommilitonen besser über die Arbeit des Fachschaftsrates informieren kann. Es kommt zu keinem Ergebnis, da alle Ansätze bereits einmal getestet wurden, letztlich aber an dem allgemeinen Desinteresse an (Hochschul-)Politik scheitern.

\section{Vollversammlung}
Es wird festgestellt, dass die Vollversammlung der Fachschaft Informatik an der Hochschule Trier im Jahr 2015 vor der konstituierenden Sitzung des zukünftigen Fachschaftsrates stattfinden muss, da nur der aktuelle Rat einen Rechenschaft ablegen kann. Es muss ein günstiger Zeitrahmen gefunden werden, um möglich viele Kommilitonen zu erreichen.\\
Es wird sich darauf geeinigt, den Termin wieder in der Mittagspause zu wählen, da am späten Abend niemand mehr kommt. Es wird weiter beschlossen, eine ansehnliche, professionelle und informative Präsentation vorzubereiten. Philipp Dippel erklärt sich dazu bereit, die Präsentation zu erstellen.\\
Jedes Ratsmitglied wird dazu angewiesen, zusammenzufassen, was es über seine Arbeit im letzten Jahr im Fachschaftsrat mitteilen kann.

\section{Sonstiges}

\subsection{Austritte}
Es geht zu Protokoll, dass Dominik Petersdorf und Tim Grundmanns aus dem Fachschaftsrat Informatik ausgetreten sind, da Sie nun Ämter im Allgemeinen Studierendenausschuss und dem Studierendenparlament bekleiden.

\subsection{Stromanschluss im Außenbereich}
Michael Ochmann gibt einen Hinweis von Arne Schmitt zu Protokoll, dass währen des Umbaus der Straßen auf dem Campus ein Stromkabel im Grünbereich am L-Gebäude vergraben wurde, um dort eine Außensteckdose anzubringen. Diese Steckdose wurde immer noch nicht gesetzt.\\
Georg Schäfer wird Alexander Föhr darauf ansprechen.

\subsection{Fahrradständer am L-Gebäude}
Der Fachschaft Informatik wurde vom ehemaligen Dekan des Fachbereichs, Herrn Professor Doktor Andreas Künkler ein Fahrradständer am L-Gebäude versprochen. Dieser Fahrradständer ist ebenfalls noch nicht vorhanden.\\
Georg Schäfer wird den Dekan, Herrn Professor Doktor Rainer Oechsle darauf ansprechen.

\subsection{Renovierung der Bestuhlung in Hörsaal 3}
Santo Pfingsten teilt mit, dass die versprochene Erneuerung der Bestuhlung von Hörsaal 3 noch in diesem Semester stattfinden wird. Die dort ansässigen Veranstaltungen werden für den Renovierungszeitraum in die Aula verlegt.

\subsection{alte lack tische}
müssen weg

\subsection{Pi and More}
fsri soll nächste pi \& more (sa 16. januar) an der hst veranstalten. räume sind beantragt, warten auf repsonse.

ausschuss soll gegründet werden, wg unabhängigkeit vom fsri wg neuwahlen etc.

\abstimmung{Es wird der Antrag gestellt, einen Ausschuss zu bilden der mit entsprechenden Vollmachten ausgestattet ist um unabhängig vom amtierenden FSRI die Veranstaltung zu organisieren.}{8}{0}{0}

Beschluss \#FSRI\_4 dass das so ist.

(Mitglieder mike, georg, santo)

pis sollen organisiert werden. 
Pies sollen organisiert werden (kuchen)[lustige witz von santo]

\subsection{Raum}
putzplan funzt nicht. plan abgelaufen. albsi neu machen?

tische leicht lädiert.

santo bemerkt dreckiges besteck und müll.

raumzugang auf hp beantragen. 

\abstimmung{Es wird der Antrag gestellt, die LACK Tische zu entsorgen}{8}{0}{0}
Es ergeht der Beschluss \textit{(\#FSRI\_3)} dass die LACK Tische entsorgt werden.

\subsection{GDW Orga}
mike -> fachseminarteilnehmer als haupt orga leute ungeeignet. jemand "greifbares" mit feuer und flamme.
kritik an santo per mail.

Georg beendet die sitzung um 20:58 Uhr.

% % ENDBLATT
\pagebreak
\section{Bestätigung des Protokolls}
Der Sprecher des Fachschaftsrates Informatik sowie der Protokollführer dieses Protokolls bestätigen mit Ihrer Unterschrift unter diesem Protokoll, dass selbiges inhaltlich korrekt ist und alle darin aufgeführten Beschlüsse so wie beschrieben vom Rat getragen werden.
\\

\vspace{3.5cm}
\hrulefill \hfill \hrulefill

\fsiPresident \hfill \protokoller

{\footnotesize (Sprecher)\hfill (Protokollführer)}
\end{protokoll}
