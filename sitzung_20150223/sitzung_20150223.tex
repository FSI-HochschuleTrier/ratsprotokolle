\documentclass[a4paper, 11pt]{article} % das Papierformat zuerst
\usepackage[utf8]{inputenc}
\usepackage{geometry}
\geometry{a4paper, top=25mm, left=35mm, right=25mm, bottom=25mm}
\usepackage{graphicx}
\usepackage{color}
\usepackage{epstopdf}
\usepackage[T1]{fontenc}
\usepackage{setspace}
\usepackage{tabularx}
\usepackage{blindtext}
\usepackage[ngerman]{babel} %deutsche Silbentrennung
\usepackage{titlesec}
\usepackage{enumitem} 
\usepackage{ifthen}
\usepackage{spverbatim}

\definecolor{fsi}{RGB}{0,81,150}

\newcommand{\abstimmung}[4]{
	\newcounter{summe}
	\setcounter{summe}{#3}
	\addtocounter{summe}{#4}
	\begin{flushleft}
		#1\\
	Es wird über den Antrag abgestimmt.
	\end{flushleft}
	\ifthenelse{\equal{#3}{0}\AND\equal{#4}{0}}{
	\begin{center}
		\textbf{Der Antrag wird einstimmig angenommen.}
	\end{center}
	}{
	\begin{center}
		#2 \ifthenelse{\equal{#2}{1}}{Stimme}{Stimmen} dafür, #3 \ifthenelse{\equal{#3}{1}}{Stimme}{Stimmen} dagegen, #4 \ifthenelse{\equal{#42}{1}}{Enthaltung}{Enthaltungen}\\
		\ifthenelse{#2>\value{summe}}{
		\textbf{Der Antrag ist somit angenommen.}
		}{
		\textbf{Der Antrag ist somit abgelehnt.}
		}
	\end{center}
	}  
}


% VARIABLEN
\newcommand{\protokoller}{Elias Broschin}
\newcommand{\dateOfMeeting}{23. Februar 2015}
\newcommand{\TeXer}{Tobias Meier}
\newcommand{\fsiPresident}{Georg Schäfer}


\begin{document}
%deckblatt start

\doublespacing
\thispagestyle{empty}

\begin{center}
\includegraphics[width=10.0cm]{../logo_faculty_computer_science.eps}

\vspace*{\fill}
{\LARGE \textbf{Protokoll der Sitzung des Fachschaftsrates \\vom \dateOfMeeting}}\\
Fachschaftsrat Informatik\\
Trier University of Applied Sciences\\
\vspace{2.5cm}
\textit{
	Protokollführer: \textbf{\protokoller} \\
	\LaTeX - Umsetzung von \TeXer\\
	am \today
}
\vfill
\end{center}

\hspace*{-35cm}
\textcolor{fsi}{\rule{64.9cm}{15pt}}
\pagebreak
 
\setcounter{tocdepth}{2}
\tableofcontents 
\pagebreak

\section{Eröffnung}
Als Protokollführer wird \protokoller~bestimmt.\\
Der Sprecher des Fachschaftsrates \fsiPresident~eröffnet die Sitzung um 19:17 Uhr.
\\\\
%\textbf{Es wird festgestellt, dass der Fachschaftsrat vollzählig und beschlussfähig ist.}
%\\\\
%\\\\
\textbf{Es wird festgestellt, dass der Fachschaftsrat beschlussfähig ist.}\\
\textbf{Es fehlen:} Benjamin Albsmeier, Marco Lochen 
%\\\\
%% Das Protokoll der letzten Sitzung wurde verlesen und genehmigt.
%Die Genehmigung des Protokolls der letzten Sitzung wird auf die nächste Sitzung vertagt.

%\abstimmung{Es wird der Antrag gestellt, Dinge zu tun}{5}{6}{2} -- {dafür}{dagegen}{Enthaltungen}
\section{Finanzen}
In der Ratssitzung am 14. Januar 2015, wurde beschlossen einen neuen Server anzuschaffen. Hierzu wurde ein Antrag beim AStA eingereicht, welcher wegen Unklarheiten vom Allgemeinen Studierendenausschuss abgelehnt wurde.\\
Der Antrag wird vom Fachschaftsrat ergänzt und dem AStA zur nochmaligen Abstimmung vorgelegt.\\
Zur Ausführung seiner Pflichten benötigt der Fachschaftsrat einen Drucker. Aus diesem Grund wurde ein enstsprechender Antrag beim AStA gestellt. Dieser Antrag wurde vom AStA genehmigt.\\
Der Antrag auf den 3D-Druck der Buzzergehäuse wurde dem AStA vorgelegt und ebenfalls abgelehnt.\\
Es wurde über die Anschaffungsliste, welche man schon länger machen wollte (siehe Protokoll 14. Januar 2015), gesprochen. Da sich allerdings niemand damit befasst hat, ist man übereingekommen diese bis zur nächsten Sitzung nochmals zu ergänzen bzw. zu überarbeiten.\\

\section{Anschaffungen}
Die Sofas im Fachschaftsraum wurden von Johannes Kirchner und Marco Lochen notdürftig repariert. Da dies allerdings lediglich eine Notlösung ist, will sich der Fachschaftsrat über die Kosten der Anschaffung neuer Sitzgelegenheiten informieren.\\
\newpage
Es soll nach 2 Sofas mit folgenden Eigenschaften Ausschau gehalten werden: 
\begin{itemize}
	\item Bezug abziehbar und Waschmaschinen verträglich
	\item 3er Sofa
\end{itemize}
Die "`LACK"'-Tische im Fachschaftsraum sind ebenfalls zum Großteil defekt. Daher wurde vereinbart auch Angebote für kleine Tische herauszusuchen.
Es wurde angemerkt, dass zu präferieren sei, wenn diese Tische abgerundete Kanten hätten.
Philipp Dippel wird sich über Angebote für die beiden Anschaffungen informieren. Die Ergebnisse dieser Suche werden dem Rat in der nächsten Sitzung mitgeteilt.\\\\
Benjamin Albsmeier tritt der Sitzung um 19:28 Uhr bei.

\section{Events}
\subsection{Bits don't Bite}
Die neue Form in der die \textit{"`Bits don't Bite"'} durchgeführt wurde hat die Besucherzahlen nicht nennenswert gehoben. Es wird festgesetzt, dass das Konzept zur Durchführung bis zu der nächsten BdB überarbeitet wird.

\subsection{Kneipentour}
Der Termin für die Kneipentour wird auf \textbf{Mittwoch, den 15.04.2015} festgesetzt. Eine Bekanntmachung des Termins per E-Mail steht noch aus.\\
Die Gaststätten werden von allen Gruppen in einer festgelegten Reihenfolge abgelaufen.\\
Es wird überlegt ein Doodle zu erstellen. In diesem sollen sich interessierte Studierende ab dem 2. Semester eintragen, damit die Teilnehmerzahl besser eingeschätzt werden kann.

\subsection{Game Development Week}
Die Räume für die Durchführung der GDW sind reserviert.\\
Eine Genehmigung zum Erhalt eines Transponders zum schließen der Gebäude "`L"' und "`X"' liegt vor. Dieser muss noch beim Hausmeister abgeholt werden.\\
Zwei Kommilitonen werden die Möglichkeit wahrnehmen, die GDW als Fachseminar zu absolvieren.\\
Am Freitag vor der GDW findet ein "`Bootcamp"' statt, um wichtiges Handwerkszeug an die Teilnehmer bereits im Vorraus weiterzugeben.\\
Es werden Überlegungen angestellt, ob während der GDW ein Getränkeverkauf stattfinden kann. Von der GDW-Orga wird geprüft, in welchem Umfang dies möglich ist und ob der Vorschlag noch zeitnah beim AStA vorgelegt werden kann.

\subsection{Fußballturnier}
Das Fußballturnier soll nächstes Semester stattfinden.

\subsection{Grillen (100-Tage-Projekt)}
Im Rahmen des Projektes "`100 Tage"' der Hochschule Trier soll ein Grillen für die Erstsemester des Sommersemester 2015 stattfinden. Dieses wird vom Fachschaftsrat organisiert und veranstaltet. Für die Finanzierung ist das 100-Tage Projekt verantwortlich. \\
Fleisch soll beim C+C Schaper eingekauft werden, Getränke bekommt man vom SWT.

\subsection{Night of the living Devs}
die "`Night of the living Devs"' soll häufiger im Semester stattfinden. Es werde drei bis vier festen Termine im Semester angepeilt. Eine Organisation dieser Termine steht noch aus.

\section{Fachschaftsraum}
Der Putzplan hat während dem Semester gut funktioniert. Es soll allerdings auch ein Putzplan für die Semesterferien erstellt werden. \\
Es ergeht eine Beschwerde über ungespülte und weit verteilte Tassen auf der Arbeitsfläche im Eingangsereich.\\
Der Kaffee im Vorratsschrank ist abgelaufen. Es wird überlegt, ob man diesen verschenken oder wegwerfen soll. Eine Entscheidung wird nicht getroffen.\\
Es wird weiter angemerkt, dass die Kaffeemaschine im Fachschaftsraum defekt ist.\\
Es wird angemerkt, dass die Pflanzen im Fachschaftsraum häufiger gegossenwerden müssen.\\\\
Stefan Bodenschatz tritt der Sitzung um 19:59 Uhr bei.

\section{Webseite}
Auf der Webseite befindet sich zu wenig Content zu Events, etc.\\
Die News auf der Webseite sollen auch automatisch auf Twitter und Facebook gepostet werden. Dies soll in nächster Zeit realisiert werden. Wegen den Zugangsdaten kann Ramona angefragt werden.\\
Der Propagandaminister, Marco Lochen, soll mehr Beiträge verfassen.

\section{Sonstiges}
Aufgrund der Kritik am Modul Angewandte-Mathematik wurde eine Mail vom Fachschaftsrat an Herrn Professor Doktor Nicolai Rudolph gesendet. Der Inhalt dieser Korrespondenz befindet sich im Anhang an dieses Protokoll.\\
Jerry würde gerne eine Spendenkasse zum gemeinschaftlichen Kaufen von Tee im Fachschaftsraum aufstellen. Der Fachschaftsrat darf dies (nach unserem Wissen) nicht. Es wurde Jerry gestattet, in eigeninitiative eine Kasse aufzustellen, allerdings muss klar ersichtlich sein, dass diese nicht vom Fachschaftsrat Informatik platziert wurde und wozu die Geldmittel verwendet werden.\\\\
\textbf{Die Sitzung wird um 20:19 Uhr geschlossen.}
\pagebreak

\section{Bestätigung des Protokolls}
Der Sprecher des Fachschaftsrates Informatik sowie der Protokollführer dieses Protokolls bestätigen mit Ihrer Unterschrift unter diesem Protokoll, dass selbiges inhaltlich korrekt ist und alle darin aufgeführten Beschlüsse so wie beschrieben vom Rat getragen werden.
\\

\vspace{3.5cm}
\hrulefill \hfill \hrulefill

\fsiPresident \hfill \protokoller

{\footnotesize (Sprecher)\hfill (Protokollführer)}

\pagebreak
\section{Anhänge}
\singlespacing
\begin{spverbatim}
Sehr geehrter Herr Schäfer,
vielen Dank für Ihre Hinweise.

Ich freue mich, dass Sie sich der Kritik angenommen haben.  Ich habe diese geprüft und bin zu der Überzeugung gelangt, dass die genannten  Punkte einer objektiven Betrachtung nicht standhalten. Die folgenden Punkte bitte ich nur als Hinweis, nicht als Entschuldigung zu verstehen:

	1. die Übungsblätter sind nummeriert. Die Reihenfolge liegt also fest.
	2. Die genannten zu wenig behandelten Verfahren wurden in der Vorlesung Schritt für Schritt vorgerechnet. Es existieren schriftliche Anleitungen. Deshalb wollten einige Übungsgruppen mehr Zeit für andere Aufgaben investieren. Diesem Wunsch bin ich nachgekommen.
	3. Es existieren so viele Übungsaufgaben, dass die Zeit nicht dafür ausreichte.
	4. Mit 25% Durchfallquote sticht die Klausur nicht übermäßig hervor.
	5. Schwierige Sonderfälle kamen in der Klausur nicht vor. 
	6. Viele Aufgaben glichen denen der letzten Klausuren.

Die Klausur wurde auch von Herrn Romes als besonders einfach klassifiziert. Wir sind beide der Überzeugung, dass wir diejenigen, welche in die Veranstaltungen gekommen sind, gut vorbereitet haben. Allerdings war die Beteiligung ab der Mitte des Semesters auch nicht besonders hoch. 108 Personen haben mitgeschrieben, doch nur ca. 50-60 kamen in die Übungen. In der letzten Vorlesung, in der typischerweise noch einmal Hinweise für die Klausur gegeben werden, kamen höchsten 75 Personen.


Sie sehen also, dass dem Eindruck entgegen die Sachlage durchaus auch  völlig anders beurteilt werden kann.

Ich bedanke mich für Ihre Mühe und 
verbleibe mit freundlichen Grüßen 
N. Rudolph


-----Ursprüngliche Nachricht----- 
Von: Georg Schäfer [mailto:schaefg@hochschule-trier.de] 
Gesendet: Dienstag, 24. Februar 2015 14:56 
An: Prof. Dr. Fritz Nikolai Rudolph 
Cc: Fachschaftsrat Informatik 
Betreff: Kritik am Modul Angewandte-Mathematik 

Sehr geehrter Herr Prof. Dr. Rudolph, 


sicherlich ist Ihnen die Kritik an der vergangenen Angewandte-Mathematik  
Klausur nicht entgangen. Der Fachschaftsrat hat aus diesem Grund unter 
den Studierenden eine Befragung durchgeführt, um den Gründen dieser 
Kritik auf den Grund zu gehen. Dabei wurden die Studierenden gefragt, 
wie sie mit Vorlesung, Übung und Klausur zurechtkamen. Im Folgenden 
werden die wichtigsten Kritikpunkte genannt, die sich aus den erhaltenen 
Rückmeldungen herauskristallisiert haben. 

Einige Studierende würden sich einen größeren Fokus auf klausurrelevante 
Themen wünschen. Besonders stachen hier DLGs höherer Ordnung und andere 
Iterrationsverfahren als das Newton-Raphson-Verfahren heraus - Diese 
Themen wurden wohl nicht ausreichend in der Veranstaltung behandelt. 
Des weiteren wurde die Organisation der Übungsaufgaben angesprochen. 
Scheinbar war oft nicht klar welche Übungsblätter zu bearbeiten waren 
und der Stand der Übungsgruppen schien starke Diskrepanzen aufzuweisen. 
Außerdem wurde bemängelt, dass zu einigen Themen zu wenige 
Übungsaufgaben existieren bzw. diese in den Übungen nicht ausreichend 
vertieft wurden. 

Wir hoffen dass Ihnen diese Kritik weiterhilft. 

Mit freundlichen Grüßen 
Georg Schäfer 

--  
Georg Schäfer 
Sprecher Fachschaftsrat Informatik 
INF | INF 

\end{spverbatim}

\end{document}
