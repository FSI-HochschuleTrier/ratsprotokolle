\documentclass[a4paper, 11pt]{article} % das Papierformat zuerst
\usepackage[utf8]{inputenc}
\usepackage{geometry}
\geometry{a4paper, top=25mm, left=35mm, right=25mm, bottom=25mm}
\usepackage{graphicx}
\usepackage{color}
\usepackage{epstopdf}
\usepackage[T1]{fontenc}
\usepackage{setspace}
\usepackage{tabularx}
\usepackage{blindtext}
\usepackage[ngerman]{babel} %deutsche Silbentrennung
\usepackage{titlesec}
\usepackage{enumitem}
\usepackage{eurosym} 
\usepackage[utf8]{inputenc}
\DeclareUnicodeCharacter{20AC}{\euro}

\definecolor{fsi}{RGB}{0,81,150}

% VARIABLEN
\newcommand{\protokoller}{Dominik Petersdorf}
\newcommand{\dateOfMeeting}{14. Januar 2015}
\newcommand{\TeXer}{Dominik Petersdorf}
\newcommand{\fsiPresident}{Georg Schäfer}
\newcommand{\fsiLeitung}{Thadeusz W{\'o}jcik}


\begin{document}
%deckblatt start

\doublespacing
\thispagestyle{empty}

\begin{center}
\includegraphics[width=10.0cm]{../logo_faculty_computer_science.eps}

\vspace*{\fill}
{\LARGE \textbf{Protokoll der Sitzung des Fachschaftsrates \\vom \dateOfMeeting}}\\
Fachschaftsrat Informatik\\
Trier University of Applied Sciences\\
\vspace{2.5cm}
\textit{
	Protokollführer: \textbf{\protokoller} \\
	\LaTeX - Umsetzung von \TeXer\\
	am \today
}
\vfill
\end{center}

\hspace*{-35cm}
\textcolor{fsi}{\rule{64.9cm}{15pt}}
\pagebreak
 
\setcounter{tocdepth}{2}
\tableofcontents 
\pagebreak

\section{Eröffnung}
Als Protokollführer wird \protokoller~bestimmt.\\
Der Sprecher des Fachschaftsrates \fsiPresident~eröffnet die Sitzung um 19:20 Uhr.
\\
Der stellv. Sprecher \fsiLeitung~übernimmt die Leitung der Sitzung.
\\\\
\textbf{Es wird festgestellt, dass der Fachschaftsrat beschlussfähig ist.}
\\\\
Die Genehmigung des letzten Protokolls wurde auf die nächste Sitzung verschoben.

\subsection{Anwesende}
\begin{tabular}{ll}
	\textbf{Ratsmitglieder} & \textbf{zusätzlich anwesende}\\
	Georg Schäfer & Thomas Eppers\\
	Michael Ochmann & Simon Zeidler\\
	Johannes Kirchner & Stefan Bodenschatz\\
	Thadeusz W{\'o}jcik & Jeremias Bose\\
	Fabio Gimmillaro\\
	Tim Grundmanns\\
	 Marco Lochen\\
	 Philipp Dippel\\
	Tobias Meier\\
	Dominik Petersdorf\\
\end{tabular}
\vspace{1.0cm}
\section{Finanzen}

Tim Grundmanns teilt mit, dass durch das Event "''Glühen"' ein Verlust von 122,39€
\\
erwirtschaftet wurde.
\\
Weiterhin wird mitgeteilt, dass aus dem Verkauf des "'Glühen"' Restbestände verblieben sind, mit denen während des Events "'31C3"' ein Gewinn von 65,-€ erwirtschaftet wurde.
\\
Es verbleiben Restbestände im Besitz der Fachschaft.

\section{Events}


Fabio Gimmilaro berichtet über den Erfolg der "'Night of the Living Deaths"' und die geplante Wiederholung des Events während oder nach der Klausurphase.
Das "'Fußball-Turnier"' ist für das Sommersemester geplant.
\\
Hierbei teilt Fabio Gimmilaro mit, dass es bei einer Anmietung des Waldstadions keinen eigenen Getränkeverkauf, aufgrund der dortigen Hausordnung, geben kann. Aus diesem Grund wird die Turnhalle der Hochschule Trier als Verantaltungsort favorisiert.
\\\\
Dominik Petersdorf erwähnt den mäßigen Erfolg des "'Glühens"' und man ist sich einig für das nächste "'Glühen"' mehr Werbung zu machen. Weiterhin teilt er mit, das der "'Analoge Spieleabend"' gut besucht war und es wahrscheinlich nach der Klausurphase einen weiteren "'Spieleabend"' geben wird. 
\\
Für das nächste Semester werden die Studierenden zu Beginn der Vorlesungen über alle stattfindenden "'Spieleabende"' informiert. Für die "'NIV"' wurde der 16. Januar festgelegt. Hierfür wurden Räume im X-Gebäude reserviert und die Netzwerkdosen in diesen Räume durch das Rechenzentrum freigeschaltet. Hierbei teilt Johannes Kirchner mit, dass Knut als Gateway genutzt wird.
\\\\
Tobias Meier berichtet, dass der entgültige Termin der "'GDW"' Ende nächster Woche, nach der Auswertung der Umfrage festgelegt wird.
\\
Hierbei teilt Tobias noch mit, dass das Grillen dieses Semester vorraussichtlich nicht mehr stattfinden wird, da durch die anstehenden Events und die Klausurphase keine Zeit mehr für das Grillen verbleibt, wobei auch die Wetterlage unklar ist.

\section{Anschaffungen}

Es wird über den Antrag aus der letzten Sitzung zur Anschaffung eines neuen Servers gesprochen.
\\\\
Es wird der Antrag gestellt, die Anschaffung des neuen Servers erst mit der Wiederaufnahme der Wirtschaftlichkeit zu tätigen.

\begin{center}
	\textbf{Der Antrag wird mit acht Ja Stimmen und zwei Enthaltungen angenommen.}
\end{center}
\ \\
Es wird darüber gesprochen, ob neue Sofas angeschafft werden sollen, oder die bestehenden repariert werden sollen.
Hierbei stellen Johannes Kirchner und Marco Lochen sich bereit, die Sofas zu überprüfen und gegebenenfalls zu reparieren.
\\\\
Es wird über weitere Anschaffungen gesprochen, für die eine Anschaffungsliste erstellt und dem ASTA vorgelegt werden soll.

\begin{itemize}\itemsep 0pt

	\item Mäuse
	\item Tastaturen
	\item Cinch-Kabel
	\item HDMI-Hub
	\item ... 
        
\end{itemize}

\begin{flushleft}
Es wird der Antrag gestellt, eine Anschaffungsliste zu erstellen und dem ASTA zukommen zu lassen.
\end{flushleft}

\begin{center}
	\textbf{Der Antrag wird mit zehn Ja Stimmen angenommen.}
\end{center}

\section{Satzung}

Georg Schäfer teilt mit, dass die Satzung der Allgemeinen Studierendenschaft geändert wurde, daher schlägt er vor, einen Ausschuss zu bilden, um die Satzung der Fachschaft Informatik zu ändern.
\\
 Es wurde festgestellt, dass die derzeitige Satzung der Fachschaft Informatik in Teilen gegen die Satzung des Fachbereichs Informatik verstößt, auch sollen Anpassungen am Wahlsystem erfolgen. 
\\
Hierfür wird vorgeschlagen, jeweils zwei Mitglieder des Fachschaftsrates und der Studierenden der Fachschaft auszuwählen, um den Ausschuss zu bilden. Desweiteren wird über eine Aufwandsentschädigung für die Mitglieder des Ausschusses gesprochen.
\\\\
Georg Schäfer stellt den Antrag, einen Ausschuss zu bilden, der eine Änderung der Satzung der Fachschaft erstellt. 

\begin{center}
	\textbf{Der Antrag wird mit zehn Ja Stimmen angenommen.}
\end{center}
\ \\
Als Mitglieder des Fachschaftsrates Informatik schlagen sich Georg Schäfer und Tobias Meier vor.\\
Georg Schäfer soll eine E-Mail an die Studierenden der Fachschaft Informatik senden, um Mitglieder für den Ausschuss zu finden. Auf der nächsten Sitzung soll dann über die weiteren Mitglieder abgestimmt werden.

\section{Sonstiges}

Prof. Rudolph sucht Mitglieder für den Entwurf der Fragebögen über die Vorlesungen im Fachbereich Informatik. 
Diese finden jedes Semester intern statt und es werden 2-3 Studierende gesucht, um Prof. Rudolph bei deren Erstellung zu helfen.
\\
Hierfür stellen sich Georg Schäfer, Thadeusz W{\'o}jcik und Philipp Dippel bereit.
\\\\
An Mira Maier von "'Barrierefrei studieren"' soll eine E-Mail geschickt werden, ihre E-Mail bitte an den Verteiler zu schicken und einer Zusage zur Auslegung von Broschüren im Fachschaftsraum.
\\\\
Es wird über die E-Mail bezüglich des "'Battle of the Profs"' gesprochen und festgestellt, dass kein Interesse seitens des Fachschaftsrates besteht daran teilzunehmen.
\\\\
Tim Grundmanns teilt mit, dass die Prüfung der Finanzen abgeschlossen ist und der alte Rat entlastet werden kann.
\\\\
Georg Schäfer stellt den Antrag, den alten Rat zu entlasten.

\begin{center}
	\textbf{Der Antrag wird mit zehn Ja Stimmen angenommen.}
\end{center}

\ \\\
Michael Ochmann stellt den Antrag, bei der Hochschulleitung den Antrag zu stellen, das Bild von Steve Jobs von der "'Wall of Fame"' zu entfernen und zu ersetzen.

\begin{center}
	\textbf{Der Antrag wird mit neun Ja Stimmen und einer Enthaltung angenommen.}
\end{center}
\ \\
Abschließend schlägt Jeremias Bose vor den IRC der Fachschaft Informatik bekannter zu machen. Hierzu soll eine E-Mail an die Studierenden der Fachschaft Informatik gesendet werden.
\\\\
Der Sprecher Georg Schäfer beendet die Sitzung um 20:17 Uhr.

\section{Bestätigung des Protokolls}
Der Sprecher des Fachschaftsrates Informatik sowie der Protokollführer dieses Protokolls bestätigen mit Ihrer Unterschrift unter diesem Protokoll, dass selbiges inhaltlich korrekt ist und alle darin aufgeführten Beschlüsse so wie beschrieben vom Rat getragen werden.
\\
\vspace{1.5cm}



\vspace{3.5cm}
\hrulefill \hfill \hrulefill

\fsiPresident \hfill \TeXer

{\footnotesize (Sprecher)\hfill (Protokollführer)}

\end{document}
