\documentclass[a4paper, 11pt]{article} % das Papierformat zuerst
\usepackage[utf8]{inputenc}
\usepackage{geometry}
\geometry{a4paper, top=25mm, left=35mm, right=25mm, bottom=25mm}
\usepackage{graphicx}
\usepackage{color}
\usepackage{epstopdf}
\usepackage[T1]{fontenc}
\usepackage{setspace}
\usepackage{tabularx}
\usepackage{blindtext}
\usepackage[ngerman]{babel} %deutsche Silbentrennung
\usepackage{titlesec}
\usepackage{enumitem} 
\usepackage{hyperref}


\definecolor{fsi}{RGB}{0,81,150}

% VARIABLEN
\newcommand{\protokoller}{Thadeusz W{\'o}jcik}
\newcommand{\dateOfMeeting}{24. November 2014}
\newcommand{\TeXer}{Thadeusz W{\'o}jcik}
\newcommand{\fsiPresident}{Georg Schäfer}


\begin{document}
%deckblatt start

\doublespacing
\thispagestyle{empty}

\begin{center}
\includegraphics[width=10.0cm]{../logo_faculty_computer_science.eps}

\vspace*{\fill}
{\LARGE \textbf{Protokoll der Sitzung des Fachschaftsrates \\vom \dateOfMeeting}}\\
Fachschaftsrat Informatik\\
Trier University of Applied Sciences\\
\vspace{2.5cm}
\textit{
	Protokollführer: \textbf{\protokoller} \\
	\LaTeX - Umsetzung von \TeXer\\
	am \today
}
\vfill
\end{center}

\hspace*{-35cm}
\textcolor{fsi}{\rule{64.9cm}{15pt}}
\pagebreak
 
\setcounter{tocdepth}{2}
\tableofcontents 
\pagebreak

\section{Eröffnung}
Als Protokollführer wird \protokoller~bestimmt.\\
Der Sprecher des Fachschaftsrates \fsiPresident~eröffnet die Sitzung um 19:25 Uhr.
\\\\
\textbf{Es wird festgestellt, dass der Fachschaftsrat vollzählig und beschlussfähig ist.}
\subsection{zusätzlich Anwesende}
\begin{itemize}
	\item Lisa Schmidt
	\item Robert Naumann
	\item Stefan Bodenschatz
	\item Thomas Eppers
\end{itemize}

\begin{flushleft}
Es wird der Antrag gestellt, einen neue Tagesordnungspunkt mit dem Titel "`Nachnominierungen"' in die TOPliste aufzunehmen.
\begin{center}
	\textbf{Der Antrag wird einstimmig angenommen.}
\end{center}
\end{flushleft}

\section{Nachnominierungen}
Die folgenden Personen bitten darum, nachträglich in den Rat aufgenommen zu werden. Dazu ist es erforderlich, dass der Kandidat weder eine 'Nein' Stimme erhält, noch die Anzahl der Enthaltungen die Anzahl der 'Ja' Stimmen übersteigt.\\
Um Aufnahme bitten folgende Personen:
\begin{flushleft}
	\textbf{Lisa Schmidt und Robert Naumann}
\end{flushleft}
\subsection{Ergebnisse}
\begin{tabular}{|l|l|l|l|}
	\hline
	\textbf{Kandidat} & \textbf{'Ja' Stimmen} & \textbf{'Nein' Stimmen} & \textbf{Enthaltungen}\\ \hline
	Lisa Schmidt & 12 & 1 & 0\\ \hline
	Robert Naumann & 3 & 2 & 8\\ \hline
\end{tabular}
\vspace{0.5cm}
\begin{flushleft}
	Die Anforderungen für die Aufnahme wurden somit von keinem der Kandidaten erfüllt.
\end{flushleft}

\section{Finanzen}
Tim Grundmanns teilt mit, dass er Probleme hat, die Finanzunterlagen vom letzten Jahr vom Allgemeinen Studierendenausschuss ausgehändigt zu bekommen, da er keine Antwort auf seine Mails erhält.
\\
Es wird das vom AStA eingeführte System vorgestellt, welches den Fachschaften eine gewisse Finanzielle Planungsmöglichkeit geben soll. Durch dieses System wird es den Fachschaften ermöglicht, Events zu organisieren und Anschaffungen zu tätigen.
\\
Johannes Kirchner erläutert das Vorhaben, den Fachschaftsserver (Knut) durch eine neue Hardware zu ersetzen. Dies wird nach der groben Schätzung von Johannes Kirchner und Michael Ochmann etwa 1260 Euro kosten.
\\\\
Es wird der Antrag gestellt, den neuen Server zu beschaffen.
\\\\
Es wird der Antrag gestellt, die Abstimmung über den vorherigen Antrag auf die nächste Sitzung zu verschieben.
\begin{center}
	\textbf{Der Antrag wird mit einer Enthaltung und zwölf Ja Stimmen angenommen.}
\end{center}

\section{Events}
Elias Broschin berichtet über den Fortschritt  bei der Organisation des Events "`Glühen"'.
\\\\
Dominik Petersdorf erwähnt die mögliche Kollision der Events "`Glühen"' und "`Analogspieleabend"'.
Weiterhin äußert er die Absicht, für die Netzwerkinstallationsveranstaltung / das LoL Turnier von der Sparkasse oder anderen Sponsoren, kleine Gewinne zu organisieren.
Die NIV wird frühestens im Januar veranstaltet.
\\\\
Fabio Gimmillaro erwähnt, dass das Fußballturnier aufgrund der Wetter- und Finanzlage auf das nächste Semester verschoben wird.
\\\\
Der Termin der "`Night of the Living Devs"' wird auf etwa Mitte Dezember festgelegt.
\\\\
Tobias Meier berichtet, dass im Rahmen der "`GameDevWeek"' weiterhin Fachseminare angeboten werden sollen, und rechnet mit dem Stattfinden des Ideenfindungstreffens in einem Zeitraum nach der Weihnachtspause.
Es wird die Möglichkeit besprochen, wegen dem Großen Andrang im letzten Semester zwei Java-Gruppen und eine C++-Gruppe zu organisieren. 
Der Termin der GDW wird vorläufig auf den Zeitraum zwischen dem \textbf{4. bis zum 10. April 2015} gesetzt.
\\\\
Das Event "`Grillen"' wird auf das Ende der Klausurenphase angesetzt. Es wird eine Umfrage geplant, um den genauen Termin festzulegen.

\section{Website}
Michael Ochmann berichtet, dass der Fachschaftsrat nun ein Wiki unter \begin{center}
\url{http://www.fsi.hochschule-trier.de/dev/null}
\end{center} betreibt, welches zur Sammlung von Informationen über den Fachschaftsrat, die Hochschule und das Studium dienen soll.
\\\\
Weiter wird festgestellt, dass der Verleihservice auf der Fachschaftswebsite nun für den Produktiveinsatz zur Verfügung steht.
\\\\
Die E-Mail-Verteiler werden ab sofort vom Rechenzentrum verwaltet, jede Position hat einen Alias, auf dem Mails empfangen werden können.
\begin{flushleft}
\small
$\cdotp$ Elias Broschin und Dominik Petersdorf verlassen um 20:38 Uhr die Sitzung
\end{flushleft}
\begin{flushleft}
\small
$\cdotp$ Elias Broschin kehrt um 20:40 Uhr zurück.
\end{flushleft}
\begin{flushleft}
\small
$\cdotp$ Dominik Petersdorf kehrt um 20:41 Uhr zurück.
\end{flushleft}

\section{Fachschaftsraum}
Benjamin Albsmeier berichtet, dass er die Mail mit der Bitte den Raumzugang für die betroffenen Personen freizuschalten, am 1. Dezember an Dieter Jahn versenden wird.

\subsection{Zustand des Raumes}
Es wird der Antrag gestellt, einen Putzplan einzuführen.
\begin{center}
	\textbf{Der Antrag wird mit zwölf Ja Stimmen und einer Enthaltung angenommen.}
\end{center}

\subsection{Mobiliar}
Es wird die Ausbesserung der Sofas im Fachschaftsraum geplant, wobei kein konkreter Termin festgelegt wird. 
\begin{flushleft}
\small
	$\cdotp$ Fabio Gimmillaro verlässt um 20:59 Uhr die Sitzung.
\end{flushleft}

Es wird der Antrag gestellt zu überprüfen ob die Sofas reparabel sind und andernfalls neues Mobiliar zu beschaffen.
\begin{center}
	\textbf{Der Antrag wird einstimmig angenommen.}
\end{center}

\section{Einarbeitung der neuen Amtsinhaber}
Die neuen Amtsinhaber berichten, dass sie sich weitestgehend in ihre Funktionen eingearbeitet haben. 

\section{Studentische Anliegen}
In Reaktion auf eine Beschwerdemail von einem Studierenden bezüglich der Vorlesungs- und Übungsgestaltung von Prof. Dr. Klösener wird einstimmig entschieden eine E-Mail an die Studierendenschaft zu versenden, in der erläutert wird wie ein Studium und studentisches Arbeiten funktionieren.
\\
Zu einer Beschwerde von Konrad Puczynski, dass die Übungsgruppen zu den meisten Veranstaltungen überfüllt seien und teilweise keine Tutorien angeboten werden, wird entschieden die Studierendenschaft über weitere Möglichkeiten zu informieren, sich Hilfe beim Lernen zu suchen.
\\
Markus Schmieder beschwert sich über die Art und Weise, mit der Prof. Dr. Georg Schneider Bonuspunkte bei Übungen handhabt.
\begin{flushleft}
\small
	$\cdotp$ Phillipp Dippel verlässt um 21:30 die Sitzung.
\end{flushleft}
\begin{flushleft}
\small
	$\cdotp$ Michael Ochmann verlässt um 21:31 Uhr die Sitzung.
\end{flushleft}
\begin{flushleft}
\small
	$\cdotp$ Michael Ochmann kehrt um 21:32 Uhr zurück.
\end{flushleft}
\begin{flushleft}
\small
	$\cdotp$ Phillipp Dippel kehrt um 21:33 Uhr zurück.
\end{flushleft}
Es wird entschieden, Markus Schmieder zu antworten, dass die Situation sich im Vergleich zu den letzten Semestern verbessert hat.

\section{Sonstiges}
Es wird der Antrag gestellt, das Protokoll der letzten Sitzung in der vorliegenden Fassung zu übernehmen.
\begin{center}
	\textbf{Der Antrag wird einstimmig angenommen.}
\end{center}

Der Sprecher Georg Schäfer schließt die Sitzung um 21:42 Uhr.

\pagebreak
\section{Bestätigung des Protokolls}
Der Sprecher des Fachschaftsrates Informatik sowie der Protokollführer dieses Protokolls bestätigen mit Ihrer Unterschrift unter diesem Protokoll, dass selbiges inhaltlich korrekt ist und alle darin aufgeführten Beschlüsse so wie beschrieben vom Rat getragen werden.
\\
\vspace{1.5cm}


\vspace{3.5cm}
\hrulefill \hfill \hrulefill

\fsiPresident \hfill \TeXer

{\footnotesize (Sprecher)\hfill (Protokollführer)}

\end{document}
